% Options for packages loaded elsewhere
\PassOptionsToPackage{unicode}{hyperref}
\PassOptionsToPackage{hyphens}{url}
%
\documentclass[
]{article}
\usepackage{amsmath,amssymb}
\usepackage{iftex}
\ifPDFTeX
  \usepackage[T1]{fontenc}
  \usepackage[utf8]{inputenc}
  \usepackage{textcomp} % provide euro and other symbols
\else % if luatex or xetex
  \usepackage{unicode-math} % this also loads fontspec
  \defaultfontfeatures{Scale=MatchLowercase}
  \defaultfontfeatures[\rmfamily]{Ligatures=TeX,Scale=1}
\fi
\usepackage{lmodern}
\ifPDFTeX\else
  % xetex/luatex font selection
\fi
% Use upquote if available, for straight quotes in verbatim environments
\IfFileExists{upquote.sty}{\usepackage{upquote}}{}
\IfFileExists{microtype.sty}{% use microtype if available
  \usepackage[]{microtype}
  \UseMicrotypeSet[protrusion]{basicmath} % disable protrusion for tt fonts
}{}
\makeatletter
\@ifundefined{KOMAClassName}{% if non-KOMA class
  \IfFileExists{parskip.sty}{%
    \usepackage{parskip}
  }{% else
    \setlength{\parindent}{0pt}
    \setlength{\parskip}{6pt plus 2pt minus 1pt}}
}{% if KOMA class
  \KOMAoptions{parskip=half}}
\makeatother
\usepackage{xcolor}
\setlength{\emergencystretch}{3em} % prevent overfull lines
\providecommand{\tightlist}{%
  \setlength{\itemsep}{0pt}\setlength{\parskip}{0pt}}
\setcounter{secnumdepth}{-\maxdimen} % remove section numbering
\ifLuaTeX
  \usepackage{selnolig}  % disable illegal ligatures
\fi
\usepackage{bookmark}
\IfFileExists{xurl.sty}{\usepackage{xurl}}{} % add URL line breaks if available
\urlstyle{same}
\hypersetup{
  hidelinks,
  pdfcreator={LaTeX via pandoc}}

\author{}
\date{}

\begin{document}

\textbf{Dr. Reuben H. Kraft}\\
The Pennsylvania State University\\
EN - Mechanical Engineering\\
(814) 867-4570\\
Email: rhk12@psu.edu

\subsection{Professional Positions}\label{professional-positions}

\subsubsection{Academic}\label{academic}

Associate Professor of Biomedical Engineering (Courtesy), The
Pennsylvania State University. (July 2019 - Present).

Associate Professor of Mechanical Engineering, The Pennsylvania State
University. (July 2019 - Present).

Assistant Professor of Biomedical Engineering (Courtesy), The
Pennsylvania State University. (July 2016 - June 2019).

Assistant Professor of Mechanical Engineering, The Pennsylvania State
University. (August 2013 - June 2019).

\subsubsection{Government}\label{government}

Mechanical Engineer, The U.S. Army Research Laboratory, Soldier
Protection Sciences Branch. (February 1, 2009 - June 22, 2012).

Post-Doc, Oak Ridge Associated Universities at The U.S. Army Research
Laboratory, Impact Physics Branch. (May 1, 2008 - February 28, 2009).

\subsubsection{Professional}\label{professional}

Founder and Chief Engineer, BrainSim Technologies Inc. (October 2019 -
Present).

Lead Researcher of Computational Biomechanics, The Johns Hopkins
University Applied Physics Laboratory, Research and Exploratory
Development Department, Biomechanics and Injury Mitigation Systems
Group. (June 26, 2012 - June 26, 2013).

\subsection{Education}\label{education}

Post-Doctoral, U.S. Army Research Laboratory, 2009.\\
Major: Mechanics

Ph D, The Johns Hopkins University, 2008.\\
Major: Mechanical Engineering\\
Dissertation Title: Computational Modeling of Damage in Brittle
Materials

MS, The Johns Hopkins University, 2006.\\
Major: Mechanical Engineering

BS, University of Maryland, Baltimore County (UMBC), 2003.\\
Major: Mechanical Engineering

\subsection{Awards and Honors}\label{awards-and-honors}

Fellow, American Society of Mechanical Engineering. (October 2023).

Outstanding Presentation Award, Penn State Center of Neural Engineering.
(August 2023).

Best Poster Award, Penn State Center of Neural Engineering. (August
2022).

Faculty Early Career Development (CAREER) Program Award, National
Science Foundation. (2019).

PSEAS Outstanding Teaching Award, The Penn State Engineering Alumni
Society (PSEAS). (2018).

Shuman Early Career Professorship, Penn State University Department of
Mechanical and Nuclear Engineering. (2013 - 2016).

First Place Paper and Oral Presentation (presented by my graduate
student, I was senior author on paper), 13th Annual Penn State College
of Engineering Research Symposium (CERS). (April 2016).

People\textquotesingle s Choice Poster Award (presented by my student, I
was senior author on poster), 2016 (Ernst) Mach Conference. (April
2016).

Presidential Early Career Awards for Scientists and Engineers (PECASE),
The White House; Office of Science and Technology Policy. (2011).

Eagle Scout, Boy Scouts of America. (1998).

\subsection{Publications}\label{publications}

\subsubsection{Book Chapter}\label{book-chapter}

\begin{enumerate}
\def\labelenumi{\arabic{enumi}.}
\item
  Dolack, M. E., Lee, C., Ru, Y., Marghoub, A., Richtsmeier, J. T.,
  Jabs, E. W., Moazen, M., Garzón-Alvarado, D. A., \& Kraft, R. H.
  (2020). Computational Morphogenesis of Embryonic Bone Development:
  Past, Present, and Future. In \emph{Mechanobiology} (pp. 197-\/-219).
  Elsevier.
\item
  Kraft, R. H., Fielding, R. A., Lister, K., Shirley, A., Marler, T.,
  Merkle, A. C., Przekwas, A. J., Tan, X. G., \& Zhou, X. (2016).
  Modeling skeletal injuries in military scenarios. In
  \emph{Mechanobiology and Mechanophysiology of Military-Related
  Injuries: Vol. 19}. Springer Berlin Heidelberg.
\item
  Clayton, J. D., \& Kraft, R. H. (2011). Mesoscale modeling of dynamic
  failure of ceramic polycrystals. In J. J. Swab (Ed.), \emph{Advances
  in Ceramic Armor VII: Ceramic Engineering and Science Proceedings:
  Vol. 568}. John Wiley \& Sons.
  \url{https://doi.org/10.1002/9781118095256.ch21}
\end{enumerate}

\subsubsection{Conference Proceeding}\label{conference-proceeding}

\begin{enumerate}
\def\labelenumi{\arabic{enumi}.}
\item
  Hannah, T., Kraft, R. H., Martin, V., \& Ellis, S. (2023, February).
  \emph{Impact of imperfect Kolsky bar experiments across different
  scales using finite elements}. Published.
  \url{https://doi.org/10.1115/IMECE2022-96816}
\item
  Martin, V., Hannah, T., Ellis, S., \& Kraft, R. H. (2023, February).
  \emph{Towards verification and validation of modeling Dyneema using
  the embedded finite element method}. Published.
  \url{https://doi.org/10.1115/IMECE2022-96784}
\item
  Hannah, T., Kraft, R. H., Martin, V., \& Ellis, S. (2021, February).
  \emph{Implications of statistical spread to experimental analysis in a
  novel miniature Kolsky bar}. Published.
  \url{https://doi.org/10.1115/IMECE2020-23976}
\item
  Fang, Z., Ranslow, A. N., \& Kraft, R. H. (2016). \emph{Computational
  micromechanics of trabecular porcine skull bone using the material
  point method.}. \emph{Volume 3: Biomedical and Biotechnology
  Engineering}, V003T04A044; 9 pages.
  \url{https://doi.org/10.1115/IMECE2016-67748}
\item
  Motiwale, S., Subramani, V. V., Zhou, X., \& Kraft, R. H. (2016).
  \emph{Damage prediction for a cervical spine intervertebral disc}.
  \emph{Volume 3: Biomedical and Biotechnology Engineering}.
  \url{https://doi.org/10.1115/IMECE2016-67711}
\item
  Chan, A. H. W., Dhobale, A., Adewole, O., Marinov, T., Kraft, R. H.,
  Cullen, D. K., \& Serruya, M. (2016). \emph{Analysis of spontaneous
  calcium signals to infer functional connectivity within a novel
  ``living electrode'' neural construct}. 1--2.
  \url{https://doi.org/10.1109/SPMB.2016.7846870}
\item
  Ranslow, A. N., Kraft, R. H., Shannon, R., De Tomas-Medina, P.,
  Radovitsky, R., Jean, A., Hautefeuille, M. P., Fagan, B., Ziegler, K.
  A., Weerasooriya, T., Dileonardi, A. M., Gunnarsson, A., \& Satapathy,
  S. (2016). \emph{Microstructural analysis of porcine skull bone
  subjected to impact loading}. \emph{Volume 3: Biomedical and
  Biotechnology Engineering}, V003T03A057; 10 pages.
  \url{https://doi.org/10.1115/IMECE2015-51979}
\item
  Lee, C., \& Kraft, R. H. (2016, July). \emph{A coupled
  reaction-diffusion-strain model for bone growth in the cranial vault}.
  Published.
\item
  Ranslow, A. N., \& Kraft, R. H. (2016, July). \emph{The development of
  a ``fuzzy'' yield envelope for trabecular porcine skull bone using
  numerical simulations}. Published.
\item
  Motiwale, S., Eppler, W., Hollingsworth, D., Hollingsworth, C.,
  Morgenthau, J., \& Kraft, R. H. (2016). Application of neural networks
  for filtering non-impact transients recorded from biomechanical
  sensors. \emph{Proceedings of the Institute of Electrical and
  Electronic Engineers (IEEE) International Conference on Biomedical and
  Health Informatics}, 204--207.
  \url{https://doi.org/10.1109/BHI.2016.7455870}
\item
  Reddy, S. N., Fielding, R. A., Robinson, M. J., \& Kraft, R. H.
  (2015). \emph{A computational study of fracture in the calcaneus under
  variable impact conditions}. \emph{Volume 3: Biomedical and
  Biotechnology Engineering}, V003T03A058; 10 pages.
  \url{https://doi.org/10.1115/IMECE2015-51984}
\item
  Kraft, R. H., \& Garimella, H. T. (2015, September). \emph{Embedded
  finite elements for modeling traumatic axonal injury}. Published.
\item
  Fielding, R. A., Tan, X. G., Przekwas, A. J., Kozuch, C. D., \& Kraft,
  R. H. (2015). \emph{High rate impact to the human calcaneus: A
  micromechanical analysis}. \emph{Volume 3: Biomedical and
  Biotechnology Engineering}, V003T03A009, (8 pages).
  \url{https://doi.org/10.1115/IMECE2014-38930}
\item
  Garimella, H. T., Yaun, H., Johnson, B. D., Slobounov, S., \& Kraft,
  R. H. (2014). \emph{Anisotropic constitutive model of human brain with
  intravoxel heterogeneity of fiber orientation using diffusion spectrum
  imaging (DSI)}. \emph{Volume 3: Biomedical and Biotechnology
  Engineering}, V003T03A011; 9 pages.
  \url{https://doi.org/10.1115/IMECE2014-39107}
\item
  Makwana, A. R., Krishna, A. R., Yuan, H., Kraft, R. H., Zhou, X.,
  Przekwas, A. J., \& Whitley, P. (2014). \emph{Towards a
  micromechanical model of intervertebral disc degeneration under cyclic
  loading}. V003T03A012; 7 pages.
  \url{https://doi.org/10.1115/IMECE2014-39174}
\item
  Lee, C., Richtsmeier, J. T., \& Kraft, R. H. (2014). \emph{A
  multiscale computational model for the growth of the cranial vault in
  craniosynostosis}. V009T12A061; 6 pages.
  \url{https://doi.org/10.1115/IMECE2014-38728}
\item
  Fielding, R. A., Kraft, R. H., Ryan, T. M., \& Stecko, T. D. (2014,
  July). \emph{A micromechanics-based simulation of calcaneus fracture
  and fragmentation due to impact loading}. Published.
\item
  Zhang, J., Merkle, A. C., Carneal, C. M., Armiger, R. S., Kraft, R.
  H., Ward, E. E., Ott, K. A., Wickwire, A. C., Dooley, C. J., Harrigan,
  T. P., \& Roberts, J. C. (2013, September). \emph{Effects of
  torso-borne mass and loading severity on early response of the lumbar
  spine under high-rate vertical loading}. Published.
\item
  Kraft, R. H., Dagro, A. M., McKee, P. J., Grafton, S. T., Vettel, J.,
  McDowell, K., Vindiola, M., \& Merkle, A. C. (2013). \emph{Combining
  the finite element method with structural network-based analysis for
  modeling neurotrauma}. 4.
\item
  Scheidler, M., Fitzpatrick, J., \& Kraft, R. H. (2011). \emph{Optimal
  pulse shapes for SHPB tests on soft materials: Vol. 1} (T. Proulx,
  Ed.; pp. 259--268). Society for Experimental Mechanics Series, Dynamic
  Behavior of Materials.
  \url{https://doi.org/10.1007/978-1-4614-0216-9_37}
\item
  Kraft, R. H., Lynch, M. L., \& Vogel, E. W. (2011).
  \emph{Computational failure modeling of lower extremities}.
  \emph{RTO-MP-HFM-207AC/323(HFM-207)}.
\item
  Clayton, J. D., \& Kraft, R. H. (2011). Mesoscale modeling of dynamic
  failure of ceramic polycrystals. \emph{Advances in Ceramic Armor VII:
  Ceramic Engineering and Science Proceedings}, \emph{32}, 237--248.
\item
  Vettel, J. M., Bassett, D. S., Kraft, R. H., \& Grafton, S. T. (2010,
  December). \emph{Physics-based models of brain structure connectivity
  informed by diffusion weighted imaging}. Published.
\item
  Gazonas, G. A., McCauley, J. W., Kraft, R. H., Love, B. M., Clayton,
  J. D., Casem, D., Dandekar, D., Rice, B., Batyrev, I., Weingarten, N.
  S., \& Schuster, B. E. (2010, November). \emph{Multiscale modeling of
  armor ceramics: Focus on AlON}. Published.
\item
  Scheidler, M., \& Kraft, R. H. (2010). \emph{Inertial effects in
  compression Hopkinson bar tests on soft materials} (C. P. Hoppel,
  Ed.). U.S. Army Research Laboratory, 1st Annual ARL Ballistic
  Technology Workshop.
\item
  Kraft, R. H., Batyrev, I., Lee, S., Rollett, A. D., \& Rice, B.
  (2010). Multiscale modeling of armor ceramics. In S. M. J. J. Swab \&
  T. Ohji (Eds.), \emph{Advances in Ceramics Armor VI: Ceramic
  Engineering and Science Proceedings: Vol. 31}. John Wiley \& Sons,
  Inc. \url{https://doi.org/10.1002/9780470944004}
\item
  Wereszczak, A. A., \& Kraft, R. H. (2003). \emph{Flexural and
  torsional resonances of ceramic tiles via impulse excitation of
  vibration: Vol. 24} (W. M. Kriven \& H. T. Lin, Eds.; 4th ed., pp.
  207--213). 27th Annual Conference on Advanced Ceramics and Composites:
  B: Ceramic Engineering and Science Proceedings.
  \url{https://doi.org/10.1002/9780470294826.ch31}
\item
  Wereszczak, A. A., \& Kraft, R. H. (2002). \emph{Instrumented Hertzian
  indentation of armor ceramics: Vol. 23} (H. T. Lin \& M. Singh, Eds.;
  3rd ed., p. 11). 26th Annual Conference on Composites, Advanced
  Ceramics, Materials, and Structures: A: Ceramic Engineering and
  Science Proceedings. \url{https://doi.org/10.1002/9780470294741.ch7}
\end{enumerate}

\subsubsection{Journal Article}\label{journal-article}

\begin{enumerate}
\def\labelenumi{\arabic{enumi}.}
\item
  Martin, V., Hannah, T., Ellis, S., \& Kraft, R. H. (2023). Using the
  embedded element finite element method to simulate impact of Dyneema
  plates. \emph{Fibers and Polymers}. Published.
  \url{https://doi.org/10.1007/s12221-023-00417-z}
\item
  Hannah, T., Kraft, R. H., Martin, V., \& Ellis, S. (published). Impact
  of imperfect Kolsky bar experiments across different scales using
  finite elements. \emph{Journal of Verification, Validation and
  Uncertainty Quantification}.
\item
  Hannah, T., Shuster, B., Baker, Z., Ellis, S., \& Kraft, R. H.
  (published). Miniature Kolsky Bar Experiment Techniques Applied to
  UHMWPE Composite Analysis. \emph{Journal of Dynamic Behavior of
  Materials}.
\item
  Zuidema, T. R., Bazarian, J. J., Kercher, K. A., Rettke, D. J.,
  Mannix, R., Kraft, R. H., Newman, S. D., Ejima, K., Steinfeldt, J. A.,
  \& Kawata, K. (2023). Longitudinal association of clinical and
  biochemical biomarkers with head impact exposure in adolescent
  football. \emph{JAMA Network Open}. Published.
  \url{https://doi.org/10.1001/jamanetworkopen.2023.16601}
\item
  Menghani, R. R., Dasans, A., \& Kraft, R. H. (2023). A sensor-enabled
  cloud-based computing platform for computational brain biomechanics.
  \emph{Computer Methods in Biomechanics and Biomedical Engineering}.
  Published. \url{https://doi.org/10.1016/j.cmpb.2023.107470}
\item
  Ramtani, S., Sánchez, J. F., Boucetta, A., Kraft, R. H.,
  Vaca-González, J. J., \& Garzón-Alvarado, D. A. (2023). A coupled
  mathematical model between bone remodeling and tumors: a study of
  different scenarios using Komarova's model. \emph{Biomechanics and
  Modeling in Mechanobiology}. Published.
  \url{https://doi.org/10.1007/s10237-023-01689-3}
\item
  Ji, S., Ghajari, M., Mao, H., Kraft, R. H., Hajiaghamemar, M., Panzer,
  M. B., Willinger, R., Gilchrist, M. D., Kleiven, S., \& Stitzel, J. D.
  (2022). Use of brain biomechanical models for monitoring impact
  exposure in contact sports. \emph{Annals of Biomedical Engineering}.
  Published. \url{https://doi.org/10.1007/s10439-022-02999-w}
\item
  Martin, V., Kraft, R. H., Hannah, T., \& Ellis, S. (2022). An
  energy-based study of the embedded element method for explicit
  dynamics. \emph{Advanced Modeling and Simulation in Engineering
  Sciences}. Published. \url{https://doi.org/10.1186/s40323-022-00223-x}
\item
  Adewole, D. O., Struzyna, L. A., Harris, J. P., Nemes, A. D., Burrell,
  J. C., Petrov, D., Kraft, R. H., Chen, I., Serruya, M. D., Wolf, J.
  A., \& Cullen, K. (2021). Development of optically controlled ``living
  electrodes'' with long-projecting axon tracts for a synaptic
  brain-machine interface. \emph{Science Advances}, \emph{7}(4).
  \url{https://doi.org/10.1126/sciadv.aay5347}
\item
  Marinov, T., Yuchi, L., Adewole, D. O., Cullen, D. K., \& Kraft, R. H.
  (2020). A computational model of bidirectional axonal growth in
  micro-tissue engineered neuronal networks (micro-TENNs). \emph{In
  Silico Biology}, \emph{13}(3-4), pp. 85--pp. 99.
  \url{https://doi.org/10.3233/ISB-180172}
\item
  Subramani, A. V., Whitley, P., Garimella, H. T., \& Kraft, R. H.
  (2020). Fatigue damage prediction in the annulus of cervical spine
  intervertebral discs using finite element analysis. \emph{Computer
  Methods in Biomechanics and Biomedical Engineering}, \emph{23}(11),
  773--784. \url{https://doi.org/10.1080/10255842.2020.1764545}
\item
  Carrera-Pinzón, A. F., Márquez-Flórez, K., Kraft, R. H., Ramtani, S.,
  \& Garzón-Alvarado, D. A. (2019). Computational model of a synovial
  joint morphogenesis. \emph{Biomechanics and Modeling in
  Mechanobiology}, 1-\/-14.
  \url{https://doi.org/10.1007/s10237-019-01277-4}
\item
  Kraft, R. H., Lee, C., Richtsmeier, J. T., \& Dolack, M. E. (2019).
  Exploring mechanisms of cranial vault development using a coupled
  turing-biomechanical model. \emph{The FASEB Journal}, \emph{33},
  326.2--326.2.
  \url{https://doi.org/10.1096/fasebj.2019.33.1_supplement.326.2}
\item
  Lee, C., Richtsmeier, J. T., \& Kraft, R. H. (2019). A coupled
  reaction--diffusion--strain model predicts cranial vault formation in
  development and disease. \emph{Biomechanics and Modeling in
  Mechanobiology}. Published.
  \url{https://doi.org/10.1007/s10237-019-01139-z}
\item
  Przekwas, A. J., Tan, X. G., Chen, Z. J., Miao, Y., Harrand, V.,
  Garimella, H. T., Kraft, R. H., \& Gupta, R. K. (2019). Biomechanics
  of blast TBI with time resolved consecutive primary, secondary and
  tertiary loads. \emph{Military Medicine}. Published.
  \url{https://doi.org/10.1093/milmed/usy344}
\item
  Garimella, H. T., Menghani, R., Gerber, J. I., Sridhar, S., \& Kraft,
  R. H. (2018). Embedded finite elements for modeling axonal injury.
  \emph{Annals of Biomedical Engineering}. Published.
  \url{https://doi.org/10.1007/s10439-018-02166-0}
\item
  Motiwale, S., Subramani, A. V., Zhou, A., \& Kraft, R. H. (2018). A
  non-linear multi-axial fatigue damage model for the cervical
  intervertebral disc annulus. \emph{Advances in Mechanical
  Engineering}, \emph{10}(6).
  \url{https://doi.org/10.1177/1687814018779494}
\item
  Dhobale, A. V., Adewole, O., Chan, A., Marinov, T., Serruya, M.,
  Kraft, R. H., \& Cullen, D. K. (2018). Assessing functional
  connectivity across 3D tissue engineered axonal tracts using calcium
  fluorescence imaging. \emph{Journal of Neural Engineering},
  \emph{15}(5). \url{https://doi.org/10.1088/1741-2552/aac96d}
\item
  Ranslow, A., Fang, Z., De Tomas, P., Gunnarsson, A., Weerasooriya, T.,
  Satapathy, S., Thompson, K. A., \& Kraft, R. H. (2018). The multiaxial
  failure response of porcine trabecular skull bone estimated using
  microstructural simulations. \emph{American Society of Mechanical
  Engineers (ASME) Journal of Biomechanical Engineering},
  \emph{140}(10). \url{https://doi.org/10.1115/1.4039895}
\item
  Garimella, H. T., Kraft, R. H., \& Przekwas, A. J. (2018). Do
  blast-induced skull flexures result in axonal deformation?. \emph{PLOS
  One}, \emph{13}(3). \url{https://doi.org/10.1371/journal.pone.0190881}
\item
  Serruya, M. D., Harris, J. P., Adewole, D. O., Struzyna, L. A.,
  Burrell, J. C., Nemes, A., Petrov, D., Kraft, R. H., Chen, H. I.,
  Wolf, J. A., \& Cullen, D. K. (2017). Engineered axonal tracts as
  "living electrodes" for synaptic-based modulation of neural circuitry.
  \emph{Advanced Functional Materials}, 1701183--1701n/a.
  \url{https://doi.org/10.1002/adfm.201701183}
\item
  Lee, C. X., Richtsmeier, J. T., \& Kraft, R. H. (2017). A
  computational analysis of bone formation in the cranial vault using a
  coupled reaction-diffusion-strain model. \emph{Journal of Mechanics in
  Medicine and Biology}, \emph{17}(4).
  \url{https://doi.org/10.1142/S0219519417500737}
\item
  Garimella, H. T., \& Kraft, R. H. (2017). A new computational approach
  for modeling diffusion tractography in the brain. \emph{Journal of
  Neural Regeneration Research}, \emph{12}(1).
  \url{https://doi.org/10.4103/1673-5374.198967}
\item
  Garimella, H. T., \& Kraft, R. H. (2016). Modeling the mechanics of
  axonal fiber tracts using the embedded finite element method.
  \emph{International Journal for Numerical Methods in Biomedical
  Engineering}, \emph{33}(5), 1--21.
  \url{https://doi.org/10.1002/cnm.2796}
\item
  Fielding, R. A., Przekwas, A. J., Tan, X. G., \& Kraft, R. H. (2015).
  Development of a lower extremity model for high strain rate impact
  loading. \emph{International Journal of Experimental and Computational
  Biomechanics}, \emph{3}(2), 161--186.
\item
  Lee, C. X., Richtsmeier, J. T., \& Kraft, R. H. (2015). A
  computational analysis of bone formation in the cranial vault in the
  mouse. \emph{Frontiers in Bioengineering and Biotechnology},
  \emph{3}(24). \url{https://doi.org/10.3389/fbioe.2015.00024}
\item
  Swab, J. J., Tice, J., Wereszczak, A. A., \& Kraft, R. H. (2014).
  Fracture toughness of advanced structural ceramics: Applying ASTM
  C1421. \emph{Journal of the American Ceramic Society}, pp. 1--pp. 9.
  \url{https://doi.org/10.1111/jace.13293}
\item
  Clayton, J. D., Kraft, R. H., \& Leavy, R. B. (2012). Mesoscale
  modeling of nonlinear elasticity and fracture in ceramic polycrystals
  under dynamic shear and compression. \emph{Journal of Solids and
  Structures}, \emph{49}(18), 6.
  \url{https://doi.org/10.1016/j.ijsolstr.2012.05.035}
\item
  Kraft, R. H., Mckee, P. J., Dagro, A. M., \& Grafton, S. T. (2012).
  Combining the finite element method with structural connectome-based
  analysis for modeling neurotrauma: Connectome neurotrauma mechanics.
  \emph{PLoS Computational Biology}, \emph{8}(8), e1002619.
  \url{https://doi.org/10.1371\%2Fjournal.pcbi.1002619}
\item
  Kraft, R. H., \& Molinari, J. F. (2008). A statistical investigation
  of the effects of grain boundary properties on transgranular fracture.
  \emph{Acta Materialia}, \emph{56}(17), 10.
  \url{https://doi.org/10.1016/j.actamat.2008.05.036}
\item
  Kraft, R. H., Molinari, J. F., Ramesh, K. T., \& Warner, D. W. (2008).
  Computational micromechanics of dynamic compressive loading of a
  brittle polycrystalline material using a distribution of grain
  boundary properties. \emph{The Journal of Mechanics and Physics of
  Solids}, \emph{56}, 23.
  \url{https://doi.org/10.1016/j.jmps.2008.03.009}
\end{enumerate}

\subsubsection{Other}\label{other}

\begin{enumerate}
\def\labelenumi{\arabic{enumi}.}
\item
  Marinov, T., Yuchi, L., Adewole, D. O., Cullen, D. K., \& Kraft, R. H.
  (published). A computational model of bidirectional axonal growth in
  micro-tissue engineered neuronal networks (micro-TENNs). In
  \emph{bioRxiv}. Cold Spring Harbor Laboratory.
  \url{https://doi.org/10.1101/369843}
\item
  Gerber, J. I., Kraft, R. H., \& Garimella, H. T. (2018). Computation
  of history-dependent mechanical damage of axonal fiber tracts in the
  brain: towards tracking sub-concussive and occupational damage to the
  brain. In \emph{bioRxiv}. \url{https://doi.org/10.1101/346700}
\item
  Garimella, H. T., Menghani, R., Gerber, J. I., Sridhar, S., \& Kraft,
  R. H. (2018). Embedded finite elements for modeling axonal injury. In
  \emph{engrXiv}. \url{https://doi.org/10.31224/osf.io/2dx5e}
\item
  Adewole, D. O., Struzyna, L. A., Harris, J. P., Nemes, A. D., Burrell,
  J. C., Petrov, D., Kraft, R. H., Chen, I., Serruya, M. D., Wolf, J.
  A., \& Cullen, K. (2018). Optically-controlled "living electrodes"
  with long-projecting axon tracts for a synaptic brain-machine
  interface. In \emph{bioRxiv}. \url{https://doi.org/10.1101/333526}
\item
  Dagro, A. M., McKee, P. J., Kraft, R. H., Zhang, T. G., \& Satapathy,
  S. S. (2013). A preliminary investigation of traumatically induced
  axonal injury in a three-dimensional (3-D) finite element model (FEM)
  of the human head during blast-loading. In \emph{Army Research
  Laboratory Technical Report (ARL-TR-6504)}.
\item
  Vettel, J., Dagro, A. M., Gordon, S., Kerick, S., Kraft, R. H., Luo,
  S., Rawal, S., Vindiola, M., \& McDowell, K. (2012). Brain
  structure-function couplings (FY11). In \emph{Army Research Laboratory
  Technical Report (ARL-TR-5893)}.
\item
  Kraft, R. H., \& Wozniak, S. L. (2011). A review of computational
  spinal injury biomechanics research and recommendations for future
  efforts. In \emph{Army Research Laboratory Technical Report
  (ARL-TR-5673)}.
\item
  Kraft, R. H., \& Dagro, A. M. (2011). Design and implementation of a
  numerical technique to inform anisotropic hyperelastic finite element
  models using diffusion-weighted imaging. In \emph{Army Research
  Laboratory Technical Report (ARL-TR-5796)}.
\item
  Clayton, J. D., \& Kraft, R. H. (2011). Mesoscale modeling of dynamic
  failure of ceramic polycrystals. In \emph{Army Research Laboratory
  Reprint (ARL-RP-328)}.
\item
  Gozonas, G. A., McCauley, J. W., Batyrev, I. G., Casem, D., Clayton,
  J. D., Dandekar, D. P., Kraft, R. H., Love, B. M., Rice, B. M.,
  Schuster, B. E., \& Weingarten, N. S. (2011). Multiscale modeling of
  armor ceramics: Focus on AlON. In \emph{Army Research Laboratory
  Reprint (ARL-RP-337)}.
\item
  Vettel, J. M., Bassett, D., Kraft, R. H., \& Grafton, S. (2010).
  Physics-based models of brain structure connectivity informed by
  diffusion-weighted imaging. In \emph{Army Research Laboratory
  Technical Reprint (ARL-RP-0355)}. U.S. Army Research Laboratory.
\item
  Wereszczak, A. A., Swab, J. J., \& Kraft, R. H. (2005). Effects of
  machining on the uniaxial and equibiaxial flexure strength of CAP3
  AD-995 Al2O3. In \emph{Army Research Laboratory Technical Report
  (ARL-TR-3617)}.
\item
  Swab, J. J., Wereszczak, A. A., Tice, J., Caspe, R., Kraft, R. H., \&
  Adams, J. (2005). Mechanical and thermal properties of advanced
  ceramics for gun barrel applications. In \emph{Army Research
  Laboratory Technical Report (ARL-TR-3417)}.
\end{enumerate}

\subsection{Presentations}\label{presentations}

\subsubsection{Invited}\label{invited}

\begin{enumerate}
\def\labelenumi{\arabic{enumi}.}
\item
  Kraft, R. H., The American Society of Mechanical Engineers (ASME)
  International Mechanical Engineering Congress \& Exposition (IMECE)
  2022, "New Trends in Medical Devices Technology," Columbus, Ohio.
\item
  Kraft, R. H., IRCOBI Pre-conference Workshop: Wearable Technologies
  for the Study of Head Injury: Applications, Challenges, and
  Opportunities, "High throughput multiscale modeling of axonal fiber
  bundles in the brain of civilian athletes and the military,"
  International Research Council on Biomechanics of Injury, Porto,
  Portugal.
\item
  Kraft, R. H. (Author), 2022 Mach Conference, "Metaverse Mechanics: How
  the metaverse will save mechanics, and how mechanics will safe the
  metaverse.," Hopkins Extreme Materials Institute, Virtual.
\item
  Kraft, R. H., Penn State Workshop: Celebrating International Research
  and Education Partnership (CIREP 2021), "International collaboration
  in mechanical engineering," Remote.
\item
  Kraft, R. H., Penn State Center for Neural Engineering Seminar,
  "On-demand, no-click brain simulations," University Park, PA.
\item
  Kraft, R. H., Mechanical Engineering Department Seminar, "The
  emergence of digital health care: a focus on the brain," East Lansing,
  Michigan.
\item
  Kraft, R. H., Society for Experimental Mechanics, "The emergence of
  digital health care and what it means for experimental mechanics: a
  focus on the brain," Reno, NV.
\item
  Kraft, R. H., 16th International Symposium on Computer Methods in
  Biomechanics and Biomedical Engineering and the 4th Conference on
  Imaging and Visualization, "Multiscale modeling of axonal fiber
  bundles in the brain," New York City, NY.
\item
  Kraft, R. H., 2019 Summer Biomechanics, Bioengineering and
  Biotransport Conference (SB3C), "History dependent damage modelling
  for axonal fiber tracts of the brain," Seven Springs, PA.
\item
  Kraft, R. H. (Author and Presenter), 11th Structural Birth Defects
  Meeting, "A coupled reaction-diffusion-strain model of bone growth in
  the cranial vault," Society for Developmental Biology, Bethesda, MD.
\item
  Kraft, R. H., Invited Speaker, 43rd Northeast Bioengineering
  Conference, "Multiscale modeling of the axonal tract level in the
  brain," New Jersey Institute of Technology (NJIT), Department of
  Biomedical Engineering, Newark, NJ.
\item
  Kraft, R. H., Soldier Protection Sciences Branch Seminar Series,
  "Recent innovations in modeling the brain," Army Research Laboratory,
  Aberdeen Proving Ground, MD.
\item
  Kraft, R. H., Invited Speaker, Engineering Science and Mechanics
  Seminar Series, "Modeling axonal fiber tracts in the brain," Penn
  State University, Department of Engineering Science and Mechanics,
  University Park, PA.
\item
  Kraft, R. H., Invited Speaker, Department of Neurosurgery Seminar
  Series, "Modeling axonal fiber tracts in the brain," University of
  Pennsylvania, Department of Neurosurgery, Philadelphia, PA.
\item
  Kraft, R. H., Invited Speaker, "Modeling axonal fiber tracts in the
  brain," Penn State University, Department of Biomedical Engineering,
  University Park, PA.
\item
  Kraft, R. H., Guest Lecture, "Modeling concussions in sports," KINES
  497D: Concussion in Athletics: From Brain to Behavior, University
  Park, PA.
\item
  Kraft, R. H., Biomedical Engineering Seminar Series, "Modeling damage
  in axonal fiber tracts," New Jersey Institute of Technology,
  Department of Biomedical engineering, University Heights Newark, New
  Jersey.
\item
  Lee, C. (Author and Presenter), Richtsmeier, J. T. (Author), Kraft, R.
  H. (Author), 1st Pan American Congresses on Computational Mechanics
  (PANACM), "A computational model for biomechanical analysis of bone
  formation in the cranial vault," International Association for
  Computational Mechanics (IACM), Buenos Aires, Argentina.
\item
  Kraft, R. H., Garimella, H. T., 1st Pan American Congresses on
  Computational Mechanics (PANACM), "Computational modeling of axonal
  injury using the embedded element approach," International Association
  for Computational Mechanics (IACM), Buenos Aires, Argentina.
\item
  Kraft, R. H., Biomedical Engineering Seminar Series, "Biomechanics of
  humans in extreme environments," Pontificia Universidad Catolica de
  Chile, Department of Structural and Geotechnical Engineering,
  Biomedical Engineering Group, Chile.
\item
  Kraft, R. H., "Damaged connectomes: A physics-based method to degrade
  brain networks," Penn State Center for Neural Engineering, University
  Park, PA.
\item
  Kraft, R. H. (Presenter), Mechanical and Nuclear Engineering Seminar
  Series, "The mechanics and response of humans in extreme
  environments," University Park, PA.
\item
  Kraft, R. H., Frontiers of Cyberscience Seminar Series, "The mechanics
  and response of humans in extreme environments," Penn State
  University, University Park, PA.
\item
  Kraft, R. H., Joint Materials/Solid Mechanics Seminar Series,
  "Connectome neurotrauma mechanics: Combining the finite element method
  with structural network-based analysis for modeling neurotrauma,"
  Brown University, Providence, RI.
\item
  Kraft, R. H., Applied Physics Laboratory Biomechanics Seminar Series,
  "Computational trauma biomechanics," The Johns Hopkins University,
  Laurel, MD.
\item
  Kraft, R. H., SIERRA Seminar Series, "Biomechanical simulations with
  Sierra Presto," Sandia National Laboratories, Albuquerque, NM.
\item
  Kraft, R. H., Exxon Mobil Research Seminar Series, "Multiscale
  modeling of brittle materials," Exxon Mobil Strategic Research Center,
  Clinton, NJ.
\item
  Kraft, R. H., Computational Solid Mechanics Laboratory Seminar Series,
  "High-fidelity computational injury biomechanics," Ecole Polytechnique
  Federale de Lausanne (EPFL), Lausanne, Switzerland.
\item
  Kraft, R. H., École Normale Supérieure, "A finite element based
  micromechanical damage model for brittle materials under compressive
  loading," Solid Mechanics Seminar Series, Paris, France.
\item
  Kraft, R. H., Impact Physics Branch Seminar Series, "Optimization of a
  dynamic hardness test methodology," U.S. Army Research Laboratory.
\end{enumerate}

\subsubsection{Uncategorized}\label{uncategorized}

\begin{enumerate}
\def\labelenumi{\arabic{enumi}.}
\item
  Kraft, R. H. (Author and Presenter), 2024 Mach Conference, "The quest
  to establish finite element brain strain as a cognitive change
  indicator," Annapolis, MD.
\item
  Reyes Kadozono, A., Kraft, R. H., Penn State Graduate Exhibition,
  "Effect of seat angle on intervertebral disc in pilots exposed to
  high-G forces," Penn State Graduate School, University Park, PA.
\item
  Menghani, R. R., Kraft, R. H., Penn State Center for Neural
  Engineering Fall 2022 Retreat, "Adding axonal fiber tractography to
  the brain simulation research platform\hspace{0pt}," Penn State Center
  for Neural Engineering, State College, PA.
\item
  Menghani, R. (Presenter), Kraft, R. H., 2023 Summer Biomechanics,
  Bioengineering and Biotransport Conference (SB3C), "Analyzing real
  world head impacts using the brain simulation research platform,"
  Vail, CO.
\item
  Huber, C. M. (Presenter), Patton, D. A., Kraft, R. H., Arbogast, K.
  B., Neurotrauma 2023, "Head kinematics and brain strain associated
  with adolescent soccer heading," Austin, TX.
\item
  Kraft, R. H. (Author and Presenter), Dasans, A., 2023 Mach Conference,
  "A scalable platform for modeling blast injuries using sensors, cloud
  computing, and machine learning," Annapolis, MD.
\item
  Hannah, T. (Author and Presenter), Kraft, R. H., Martin, V., Ellis,
  S., ASME 2022 International Mechanical Engineering Congress \&
  Exposition, "Impact of imperfect Kolsky bar experiments across
  different scales using finite elements," The American Society of
  Mechanical Engineers, Columbus, Ohio.
\item
  Menghani, R. (Co-Author), Kraft, R. H. (Author and Presenter), Dasans,
  A., ASME 2022 International Mechanical Engineering Congress \&
  Exposition, "Verification and validation of a cloud-based brain
  computing service," The American Society of Mechanical Engineers,
  Columbus, Ohio.
\item
  Kraft, R. H., Dye, C., Mackay, J. C., 2022 Society of Engineering
  Science Annual Technical Meeting, "Prediction of facial overpressure
  using body worn sensors and machine learning algorithms in military
  blast environments," Society of Engineering Science, College Station,
  Texas.
\item
  Menghani, R. R., Kraft, R. H., Penn State Center for Neural
  Engineering Fall 2022 Retreat, "The brain simulation research
  platform: A sensor-enabled automated brain injury prediction service,"
  Penn State Center for Neural Engineering, State College, PA.
\item
  Kraft, R. H., 11th European Solid Mechanics Conference (ESMC2022),
  "Prediction of facial overpressure using body worn sensors and machine
  learning algorithms in military blast environments," Galway, Ireland.
\item
  Menghani, R. (Co-Author), Kraft, R. H. (Author and Presenter), Dasans,
  A., Rawat, M., Bartsch, A., ASME 2021 International Mechanical
  Engineering Congress \& Exposition, "Cost and scalability analysis of
  a cloud-based brain computing service," The American Society of
  Mechanical Engineers, Virtual.
\item
  Hannah, T. (Author and Presenter), Kraft, R. H. (Author and
  Presenter), Martin, V. (Co-Author), Ellis, S., ASME 2020 International
  Mechanical Engineering Congress \& Exposition, "Implications of
  statistical spread to experimental analysis in a novel miniature
  kolsky bar," The American Society of Mechanical Engineers, Virtual.
\item
  Kraft, R. H. (Author and Presenter), Menghani, R. (Co-Author), ASME
  2020 International Mechanical Engineering Congress \& Exposition,
  "On-demand brain simulations for prediction of cumulative head
  trauma," The American Society of Mechanical Engineers, Virtual.
\item
  Martin, V. (Author and Presenter), Kraft, R. H. (Author), Ellis, S.,
  2020 Mach Conference, "Multiscale Modeling of Dyneema using the
  Embedded Element Method," Hopkins Extreme Materials Institute, Remote.
\item
  Menghani, R. (Author and Presenter), Kraft, R. H., ASME 2019
  International Mechanical Engineering Congress \& Exposition, "Effect
  of an advanced combat helmet on axonal injury caused by primary blast
  loading," The American Society of Mechanical Engineers, Salt Lake
  City, UT.
\item
  Kraft, R. H., ASME 2019 International Mechanical Engineering Congress
  \& Exposition, "Sensor-enabled cloud based computational modeling of
  the brain," The American Society of Mechanical Engineers, Salt Lake
  City, UT.
\item
  Subramani, V. V. (Presenter), Whitley, P. E., Garimella, H. T., Kraft,
  R. H., 2019 Summer Biomechanics, Bioengineering and Biotransport
  Conference (SB3C), "Location-wise fatigue damage prediction for the
  intervertebral disc annulus of the cervical spine," Seven Springs, PA.
\item
  Kraft, R. H., 4th International Forum on Blast Injury Countermeasures
  (IFBIC), "Mechanism-based brain models to study primary blast loading
  effects on axonal deformation: the past, present and future," McLean,
  VA.
\item
  Kraft, R. H. (Author and Presenter), Penn State Institute of the
  Neurosciences 2019 Neuro-Retreat, "Multiscale modeling of axonal fiber
  tracts in the brain," Institute of the Neurosciences, University Park,
  PA.
\item
  Kraft, R. H., American Association of Anatomists (AAA) 2019 Annual
  Meeting, "Exploring mechanisms of cranial vault development using a
  coupled Turing-biomechanical model," Orlando, FL.
\item
  Hannah, T. W. (Author and Presenter), Kraft, R. H. (Author), 2019 Mach
  Conference, "Computationally confirmed Kolsky bar: An application to
  high rate testing of non-ideal Dyneema," Hopkins Extreme Materials
  Institute, Annapolis, MD.
\item
  Dolack, M. (Author and Presenter), Lee, C., Richtsmeier, J. T., Kraft,
  R. H., ASME 2018 International Mechanical Engineering Congress \&
  Exposition, "A coupled reaction-diffusion-strain model of Mesenchymal
  stem cell differentiation into osteoblasts," The American Society of
  Mechanical Engineers, Pittsburgh, PA.
\item
  Lee, C., Dolack, M., Richtsmeier, J. T., Kraft, R. H. (Author and
  Presenter), Center for Engineering MechanoBiology (CEMB) 2018
  Mechanobiology Symposium, "A new reaction-diffusion-strain model for
  skull growth and defect formation," University of Pennsylvania NSF
  Center for Engineering MechanoBiology (CEMB), Philadelphia, PA.
\item
  Kraft, R. H. (Author and Presenter), Garimella, H. T. (Author), World
  Congress of Biomechanics, "Do blast-induced skull flexures result in
  axonal deformation?," Dublin, Ireland.
\item
  Kraft, R. H. (Author and Presenter), Garimella, H. T. (Author),
  Gerber, J. I. (Author), World Congress of Biomechanics, "Tracking
  damage in a digital brain," Dublin, Ireland.
\item
  Kraft, R. H. (Author and Presenter), Slobounov, S. (Author), 2018 Big
  Ten -- Ivy League Traumatic Brain Injury Summit, "Athlete-specific
  digital brain models to characterize every impact," Philadelphia, PA.
\item
  Hertel, Z. R. (Author and Presenter), Schumacher, S. C. (Author),
  Kraft, R. H. (Author), 2018 Mach Conference, "Failure models for soft
  materials in particle based methods," Hopkins Extreme Materials
  Institute, Annapolis, MD.
\item
  Gerber, J. I. (Author and Presenter), Garimella, T. (Author), Kraft,
  R. H. (Author), ASME 2017 International Mechanical Engineering
  Congress \& Exposition, "A computational approach to model damage in
  axonal fiber tracts of the brain," The American Society of Mechanical
  Engineers, Tampa, FL.
\item
  Hertel, Z. R. (Author and Presenter), Schumacher, S. C. (Author),
  Kraft, R. H. (Author), 2017 Mach Conference, "Development of a failure
  model for biological materials within the particle based software
  Kodiak," Hopkins Extreme Materials Institute, Annapolis, MD.
\item
  Yuchi, L. (Author and Presenter), Kraft, R. H. (Author), Keystone
  Connectomics Conference X2, "Bidirectional growth model of
  micro-tissue engineered neuronal networks (micro-TENNs)," Santa Fe,
  New Mexico.
\item
  Dhobale, A. (Author and Presenter), Kraft, R. H. (Author), Keystone
  Connectomics Conference X2, "Functional connectivity analysis of
  micro-tissue engineered neural networks," Santa Fe, New Mexico.
\item
  Fang, Z. (Author and Presenter), Ranslow, A. N. (Author), Kraft, R. H.
  (Author), ASME 2016 International Mechanical Engineering Congress \&
  Exposition, "Computational micromechanics of trabecular porcine skull
  bone using the material point method," The American Society of
  Mechanical Engineers, Phoenix, AZ.
\item
  Garimella, H. T. (Author), Kraft, R. H. (Author), ASME 2016
  International Mechanical Engineering Congress \& Exposition,
  "Validation of embedded element method in the prediction of white
  matter disruption in concussions," The American Society of Mechanical
  Engineers, Phoenix, AZ.
\item
  Garimella, H. T. (Author and Presenter), Kraft, R. H. (Author), ASME
  2016 International Mechanical Engineering Congress \& Exposition,
  "Disruption in electromechanical behavior of axonal fiber tracts
  during concussion: A multiscale modeling approach," The American
  Society of Mechanical Engineers, Phoenix, AZ.
\item
  Yuchi, L. (Author and Presenter), Kraft, R. H. (Author), Bernstein
  Conference in Computational Neuroscience, "Progress on bidirectional
  growth model of micro-tissue engineered neuronal networks
  (micro-TENNs)," Berlin, Germany.
\item
  Lee, C. (Author and Presenter), Kraft, R. H. (Author), The 12th World
  Congress on Computational Mechanics WCCM XII \& The 6th Asia-Pacific
  Congress on Computational Mechanics APCOM VI, "A coupled
  reaction-diffusion-strain model of bone growth in the cranial vault,"
  International Association for Computational Mechanics (IACM) and the
  Korean Society for Computational Mechanics (KSCM), Seoul, Korea.
\item
  Garimella, H. T. (Author and Presenter), Kraft, R. H. (Author), The
  12th World Congress on Computational Mechanics WCCM XII \& The 6th
  Asia-Pacific Congress on Computational Mechanics APCOM VI, "Modeling
  the electromechanical behavior of axonal fiber bundles," International
  Association for Computational Mechanics (IACM) and the Korean Society
  for Computational Mechanics (KSCM), Seoul, Korea.
\item
  Kraft, R. H. (Author and Presenter), Garimella, H. T. (Author), The
  12th World Congress on Computational Mechanics WCCM XII \& The 6th
  Asia-Pacific Congress on Computational Mechanics APCOM VI, "Modeling
  the mechanics of axonal fiber tracts using the embedded element
  method," International Association for Computational Mechanics (IACM)
  and the Korean Society for Computational Mechanics (KSCM), Seoul,
  Korea.
\item
  Hertel, Z. R. (Author and Presenter), Schumacher, S. C. (Author),
  Kraft, R. H. (Author), 2016 Mach Conference, "Implementation of
  viscoelasticity into the CTH marker method," Hopkins Extreme Materials
  Institute, Annapolis, MD.
\item
  Ranslow, A. N. (Author and Presenter), Kraft, R. H. (Author), 2016
  Mach Conference, "The computational characterization of the multiaxial
  failure response of trabecular skull bone," Hopkins Extreme Materials
  Institute, Annapolis, MD.
\item
  Motiwale, S. (Author and Presenter), Kraft, R. H. (Author), Penn State
  13th Annual College of Engineering Research Symposium, "Understanding
  impact forces to the brain: Neural networks based impact
  classification for head impacts in sports," Penn State Engineering
  Graduate Student Council, University Park, PA.
\item
  Motiwale, S. (Author and Presenter), Eppler, W. (Author),
  Hollingsworth, D. (Author), Hollingsworth, C. (Author), Morgenthau, J.
  (Author), Kraft, R. H. (Author), The IEEE International Conference on
  Biomedical and Health Informatics (BHI), "Application of neural
  networks for filtering non-impact transients recorded from
  biomechanical sensors," IEEE Engineering in Medicine and Biology
  Society (IEEE-EMBS), Las Vegas, NV.
\item
  Garimella, H. T. (Presenter), Kraft, R. H., ASME 2015 International
  Mechanical Engineering Congress \& Exposition, "Modeling
  electromechanical deficits in the human brain," The American Society
  Of Mechanical Engineers, Houston, TX.
\item
  Sodha, K. B. (Presenter), Kraft, R. H., ASME 2015 International
  Mechanical Engineering Congress \& Exposition, "Exploration of
  miniaturized Kolsky bar designs for testing soft material properties
  at high loading rates using finite element modeling," The American
  Society Of Mechanical Engineers, Houston, TX.
\item
  Kraft, R. H., Fielding, R. A., 1st Pan American Congresses on
  Computational Mechanics (PANACM), "Fracture networks in the human
  calcaneus due to impact loading," International Association for
  Computational Mechanics (IACM), Buenos Aires, Argentina.
\item
  Motiwale, S. (Author and Presenter), Kraft, R. H. (Author), Penn State
  Neuroscience Retreat, "Understanding impact forces to the brain:
  Neural networks based impact classification for head impacts in
  sports," Penn State Institute of the Neurosciences, University Park,
  PA.
\item
  Lee, C. X. (Author and Presenter), Richtsmeier, J. T., Kraft, R. H.,
  The Mid-Atlantic American Physics Society (APS) Meeting, "A
  computational analysis of bone formation in the cranial vault,"
  University Park, PA.
\item
  Lee, C. X. (Presenter), Richtsmeier, J. T. (Author), Kraft, R. H., The
  Mid-Atlantic American Physics Society (APS) Meeting, "A computational
  analysis of bone formation in the cranial vault," University Park, PA.
\item
  Fielding, R. A., Tan, X. G., Przekwas, A., Kraft, R. H., The
  Mid-Atlantic American Physics Society (APS) Meeting, "Finite element
  modeling of impact and injury to the lower extremity," University
  Park, PA.
\item
  Ranslow, A. N., Ziegler, K. A., Satapathy, S. S., Radovitsky, R.,
  Kraft, R. H., The Mid-Atlantic American Physics Society (APS) Meeting,
  "Microstructural analysis of porcine skull bone subjected to impact
  loading," University Park, PA.
\item
  Garimella, H. T., Kraft, R. H., The Mid-Atlantic American Physics
  Society (APS) Meeting, "Reinforced composite based modeling of axonal
  injury - A physics based approach," University Park, PA.
\item
  Garimella, H. T. (Author and Presenter), Kraft, R. H. (Author), The
  Mid-Atlantic American Physics Society (APS) Meeting, "Reinforced
  composite based modeling of axonal injury - A physics based approach,"
  University Park, PA.
\item
  Kraft, R. H., "Towards a micromechanics-based simulation of calcaneus
  fracture and fragmentation due to impact loading," Department of
  Defense, U.S. Army, Aberdeen Proving Ground, MD.
\item
  Zhang, J. (Presenter), Merkle, A. C., Carneal, C. M., Armiger, R. S.,
  Kraft, R. H., Ward, E. E., Ott, K. A., Wickwire, A. C., Dooley, C. J.,
  Harrigan, T. P., Roberts, J. C., International Research Council on
  Biomechanics of Injury, "Effects of torso-borne mass and loading
  severity on early response of the lumbar spine under high-rate
  vertical loading," Sweden.
\item
  Kraft, R. H., Dagro, A. M., McKee, P. J., Grafton, S. T., Vettel, J.,
  McDowell, K., Vindiola, M., Merkle, A. C., 11th International
  Symposium on Computer Methods in Biomechanics and Biomedical
  Engineering (CMBBE), "Combining the finite element method with
  structural network-based analysis for modeling neurotrauma," Salt Lake
  City, UT.
\item
  Kraft, R. H., A Survey of Blast Injury across the Full Landscape of
  Military Science, "Computational failure modeling of lower
  extremities," NATO-HFM-207 Panel, Halifax, Canada.
\item
  Kraft, R. H., U.S. Army Research Laboratory/U.S. Army Medical Research
  and Material Command Home on Home Workshop, "Spine modeling efforts
  and opportunities for future work," U.S Army Research Laboratory
  Research Portfolio Showcase, Aberdeen Proving Ground, MD.
\item
  Kraft, R. H., Office of the Surgeon General, "BrainAid: A smartphone
  app for field-deployable multimodal screening and detection of mild
  traumatic brain injury," San Antonio, TX.
\item
  Kraft, R. H., Neurodiagnostic for the Battlefield, "BrainAid: A
  smartphone app for field-deployable multimodal screening and detection
  of mild traumatic brain injury," U.S. Medical Research and Materiel
  Command, Fort Detrick, MD.
\item
  Kraft, R. H., Department of Defense/Department of Energy Neural
  Restoration Workshop at the Center for Neurotechnology Studies of the
  Potomac Institute for Policy Studies, "A structural mechanics-based
  approach for predicting neural deficits," Arlington, VA.
\item
  Cullen, D. K. (Presenter), Kraft, R. H., Biomedical Engineering
  Society Annual Meeting, "Macro- to micro- biomechanics of traumatic
  brain injury," Austin, TX.
\item
  Kraft, R. H., U.S. Army Research Laboratory\textquotesingle s
  Accelerative Injury Workshop, "A finite element-based comparative
  study between high rate accelerative and blast-induced head trauma,"
  Aberdeen Proving Ground, MD.
\item
  Kraft, R. H., U.S. Army Research Laboratory\textquotesingle s
  Accelerative Injury Workshop, "Development of a computational
  framework for high rate injury biomechanics of lower extremities,"
  Aberdeen Proving Ground, MD.
\item
  Kraft, R. H., Defense and Veterans Brain Injury Center/Combat Casualty
  Care Research Program of the U.S. Army Medical Research and Materiel
  Command/National Institutes of Neurological Disorders and Stroke of
  the National Institute of Health at the Advanced Technology
  Applications for Combat Casualty Care Conference, "Recommendations for
  a field deployable diagnostic device for mild traumatic brain injury,"
  St. Pete\textquotesingle s Beach, FL.
\item
  Cullen, D. K. (Presenter), Kraft, R. H., Neurotrauma Society Annual
  Meeting, "Determining trauma-specific neuropathology based on macro-
  to micro-injury biomechanics," Las Vegas, NV.
\item
  Kraft, R. H., 1st Annual Ballistic Protection Technologies Workshop,
  "High rate computational brain injury biomechanics: Linkages with
  simulation-based neurophysiology," Rockville, MD.
\item
  Kraft, R. H., The Technical Cooperation Program (TTCP), "High fidelity
  computational injury biomechanics," The Defence Science and Technology
  Laboratory (DSTL), Porton Down, UK.
\item
  Kraft, R. H., Department of Defense Brain Injury Computational
  Modeling Expert Panel Meeting, "High rate computational brain injury
  biomechanics: Linkages with simulation-based neurophysiology,"
  Chantilly, VA.
\item
  Kraft, R. H., American Ceramics Society International Conference and
  Exposition, "Multiscale modeling of armor ceramics," Daytona Beach,
  FL.
\item
  Kraft, R. H., Neural Restoration Workshop, "A structural
  mechanics-based approach for predicting neural deficits," Potomac
  Institute for Policy Studies, Albuquerque, NM.
\item
  Kraft, R. H., 8th World Congress on Computational Mechanics, "A
  micromechanics-based multiscale approach for simulating dynamic crack
  propagation," Lido Island, Venice, Italy.
\item
  Kraft, R. H., American Physics Society Topical Group on Shock
  Compression of Condensed Matter, "A parallel multiscale model for
  brittle materials using a finite element based micromechanical model
  and homogenization theory," Kona, Hawaii.
\item
  Kraft, R. H., American Ceramics Society International Conference and
  Exposition, "Macroscopic measures of strength and damage computed from
  physically-based mechanisms at the micro-level," Cocoa Beach, FL.
\item
  Kraft, R. H., 9th U.S. National Congress on Computational Mechanics,
  "Finite element based modeling of damage in brittle materials: From
  micro to macro," San Francisco, CA.
\item
  Kraft, R. H., 17th US Army Symposium on Solid Mechanics, "Finite
  element based micromechanical modeling of brittle materials under
  compressive loading," Baltimore, MD.
\item
  Kraft, R. H., American Society of Mechanical Engineers International
  Congress, "A finite element based micromechanical damage model for
  brittle materials under compressive loading," Orlando, FL.
\item
  Kraft, R. H., American Ceramics Society International Conference and
  Exposition, "A computational framework for intergranular and cleavage
  fracture," Daytona Beach, FL.
\item
  Kraft, R. H., American Society of Mechanical Engineers International
  Congress, "Controlling microcracking events in ceramics: A grain
  boundary engineering approach," Orlando, FL.
\item
  Kraft, R. H., MRS Fall Meeting, "A numerical model for intergranular
  and cleavage fracture in ceramic materials," Boston, MA.
\end{enumerate}

\subsection{Contract, Fellowships, Grants and Sponsored
Research}\label{contract-fellowships-grants-and-sponsored-research}

Kraft, R. H. (Principal Investigator), "Development of Predictive Disc
Degeneration Simulations for Pilots," Sponsored by Air Force Research
Laboratory, Federal Agencies, \$359,796.00. (September 30, 2022 -
September 30, 2025).

Kraft, R. H. (Principal Investigator), "Examining the link between
finite element-based strain predictions and cognitive changes.,"
Sponsored by Chuck Noll Foundation {[}MP{]}, Nonprofit Foundations,
\$77,589.00. (August 1, 2022 - August 1, 2024).

Kraft, R. H. (Principal Investigator), "Elucidating High Strain Rate
Deformation Mechanisms in Penetration-Resistant Composites," Sponsored
by Triad National Security, LLC (was LANL - Los Alamos National
Laboratory), Federal Laboratories, \$214,423.00. (June 16, 2022 - June
30, 2024).

Kraft, R. H. (Principal Investigator), "Unfunded Collaborative Research
Agreement - Sports \& Wellbeing Analytics," Sponsored by Sports \&
Wellbeing Analytics, Corporations, \$1.00. (March 17, 2021 - March 16,
2024).

Hill, K., Kraft, R. H. (Principal Investigator), "CAREER: Multiscale
Modeling of Axonal Fiber Bundles in the Brain," Sponsored by National
Science Foundation, Federal Agencies, \$396,514.00. (February 15, 2019 -
January 31, 2024).

Hillman, M. (Principal Investigator), Kraft, R. H. (Co-Principal
Investigator), Warn, G. P. (Principal Investigator), "STTR PHASE II
Enhancing Thermo-Mechanically Coupled Computational Models for
High-Temperature Impact and Fracture," Sponsored by Karagozian \& Case,
Inc., Corporations, \$179,000.00. (July 1, 2021 - December 2, 2023).

Kraft, R. H. (Principal Investigator), "Occupational mTBI from repeated
exposure to low-level blast," Sponsored by Biokinetics and Associates
Ltd, Corporations, \$111,000.00. (May 1, 2022 - March 31, 2023).

Kraft, R. H. (Principal Investigator), "Development of a Novel Ballistic
Armor Concept using FEM," Sponsored by Triad National Security, LLC (was
LANL - Los Alamos National Laboratory), Federal Laboratories,
\$56,363.00. (July 24, 2017 - March 31, 2022).

Kraft, R. H. (Principal Investigator), "An Exploration of the Material
Point Method (MPM) in CTH Applied to Soft Material Systems Subjected to
Dynamic Loading (Continuation)," Sponsored by Sandia National
Laboratories, Federal Laboratories, \$50,000.00. (February 2, 2017 -
December 31, 2021).

Kraft, R. H. (Principal Investigator), "Development of Predictive Disc
Degeneration Simulations for Pilots," Sponsored by Air Force Research
Laboratory, Federal Agencies, \$39,315.00. (February 22, 2021 - August
21, 2021).

Kraft, R. H. (Principal Investigator), "Head Kinematics Experimentation
and Data Analysis," Sponsored by SURVICE Engineering Company, LLC,
Corporations, \$5,000.00. (September 1, 2020 - May 31, 2021).

Richtsmeier, J. T. (Principal Investigator), Drew, P. (Co-Principal
Investigator), Kraft, R. H. (Co-Principal Investigator),
"Craniosynostosis Network (formerly award number 0254-3543-4609),"
Sponsored by Icahn School of Medicine at Mount Sinai, Universities and
Colleges, \$322,692.00. (February 1, 2016 - January 31, 2021).

Niyibizi, C. (Co-Investigator), Kraft, R. H. (Co-Investigator),
Szczesny, S. (Principal Investigator), Wong, P. K. (Co-Investigator),
"Stem Cell Mechanotransduction with Tendon Fatigue," Sponsored by
University of Pittsburgh, Universities and Colleges, \$50,335.00. (July
1, 2019 - June 30, 2020).

Kraft, R. H. (Principal Investigator), "SBIR Phase II: Global-Local
Modeling of Aircraft Occupant Safety Assessment during Ejection (Air
Force Phase II SBIR)," Sponsored by CFD Research Corporation,
Corporations, \$139,494.00. (October 25, 2017 - January 20, 2020).

Kraft, R. H. (Principal Investigator), "Development of Commercial Tools
for Brain Modeling," Sponsored by CFD Research Corporation,
Corporations, \$70,637.00. (November 15, 2017 - September 15, 2019).

Niyibizi, C. (Co-Investigator), Kraft, R. H. (Co-Investigator),
Szczesny, S. (Principal Investigator), Wong, P. K. (Co-Investigator),
"Stem Cell Mechanotransduction with Tendon Fatigue," Sponsored by
University of Pittsburgh, Universities and Colleges, \$49,666.00. (July
1, 2018 - June 30, 2019).

Kraft, R. H. (Principal Investigator), "NEUP: Multilayer Composite Fuel
Cladding for LWR Performance Enhancement and Severe Accident Tolerance,"
Sponsored by Massachusetts Institute of Technology, Universities and
Colleges, \$50,000.00. (October 1, 2015 - June 30, 2019).

Kraft, R. H. (Principal Investigator), "Embedded Finite Elements for a
Multiscale, Multifunctional Approach for Modeling Axonal Bundles,"
Sponsored by SURVICE Engineering Company, LLC, Corporations,
\$114,460.00. (December 13, 2017 - March 13, 2019).

Kraft, R. H. (Principal Investigator), "Biological Living Electrodes
Using Tissue Engineered Axonal Tracts to Probe and Modulate the Nervous
System (Previously Agreement \#569770)," Sponsored by Pennsylvania,
University of, Universities and Colleges, \$124,527.00. (August 1, 2017
- July 31, 2018).

Kraft, R. H. (Principal Investigator), "STTR Phase I: Synchronizing
Video Imagery with Wearable Sensor Data and Side-by-Side Modeling
Software to Develop Healthy Habits in Children," Sponsored by CoachSafe
PlaySafe, LLC, Corporations, \$132,750.00. (July 1, 2016 - June 30,
2018).

Richtsmeier, J. T. (Principal Investigator), Drew, P. (Co-Principal
Investigator), Kraft, R. H. (Co-Principal Investigator), Rizk, E. B.
(Co-Principal Investigator), "Craniosynostosis Network," Sponsored by
Icahn School of Medicine at Mount Sinai, Universities and Colleges,
\$374,032.00. (February 1, 2015 - January 31, 2018).

Kraft, R. H. (Principal Investigator), "Embedded Finite Elements for a
Multiscale, Multifunctional Approach for Modeling Axonal Bundles,"
Sponsored by IAP Worldwide Services, Inc., Corporations, \$107,684.00.
(October 1, 2016 - September 24, 2017).

Kraft, R. H. (Principal Investigator), "Microstructural Analysis of
Porcine Skull Bone Subjected to Impact Loading," Sponsored by
Massachusetts Institute of Technology, Universities and Colleges,
\$98,000.00. (July 1, 2014 - September 1, 2017).

Kraft, R. H. (Principal Investigator), "Biological Living Electrodes
Using Tissue Engineered Axonal Tracts to Probe and Modulate the Nervous
System (Previously Agreement \#568000)," Sponsored by Pennsylvania,
University of, Universities and Colleges, \$122,898.00. (August 1, 2016
- July 31, 2017).

Kraft, R. H. (Principal Investigator), "SBIR: Phase II: A Neck Injury
Assessment Tool for Prolonged Wear of Head Supported Mass," Sponsored by
CFD Research Corporation, Corporations, \$69,086.00. (April 21, 2015 -
June 14, 2017).

Kraft, R. H. (Principal Investigator), "SBIR Phase II: Physics and
Physiology Based Human Body Model of Blast Injury and Protection,"
Sponsored by CFD Research Corporation, Corporations, \$100,000.00.
(April 1, 2015 - May 31, 2017).

Kraft, R. H. (Principal Investigator), "Global-Local Modeling of
Aircraft Occupant Safety Assessment during Ejection (Air Force SBIR
Phase I)," Sponsored by CFD Research Corporation, Corporations,
\$22,000.00. (August 4, 2016 - April 15, 2017).

Kraft, R. H. (Principal Investigator), "An Exploration of the Material
Point Method (MPM) in CTH Applied to Soft Material Systems Subjected to
Dynamic Loading," Sponsored by Sandia National Laboratories, Federal
Laboratories, \$190,644.00. (January 16, 2015 - December 31, 2016).

Kraft, R. H. (Principal Investigator), "Biological Living Electrodes
Using Tissue Engineered Axonal Tracts to Probe and Modulate the Nervous
System," Sponsored by Pennsylvania, University of, Universities and
Colleges, \$120,313.00. (September 30, 2015 - July 31, 2016).

Kraft, R. H. (Principal Investigator), "A Neck Injury Assessment Tool
for Prolonged Wear of Head Supported Mass," Sponsored by CFD Research
Corporation, Corporations, \$18,568.00. (January 15, 2014 - August 14,
2014).

Kraft, R. H. (Principal Investigator), "Physics and Physiology Based
Human Body Model of Blast Injury and Protection," Sponsored by CFD
Research Corporation, Corporations, \$36,000.00. (January 7, 2014 -
August 6, 2014).

\subsection{Intellectual Property}\label{intellectual-property}

Kraft, R. H. "Brain Simulation Technology." (Application: 2017).

Kraft, R. H. "SmartGear: Instrumented Wrestling Headgear using Sensors."
(Application: 2017).

Kraft, R. H. "Method and Apparatus for Multimodal Mobile Screening to
Quantitatively Detect Brain Function Impairment." (Application:
September 2011).

\subsection{Directed Student Learning}\label{directed-student-learning}

\subsubsection{Master\textquotesingle s Thesis}\label{masters-thesis}

\begin{enumerate}
\def\labelenumi{\arabic{enumi}.}
\item Fournier, N.,
 MS. Finite element modeling of gasket interfaces. (November 2023 - Present).
\item Lovett, J.,
 MS. Energy based body armor design. Date Graduated: August 2023. (August 2021 - August 2023).
\item Norris, I., 
MS. Computational modeling of spinal degeneration in F35 pilots. (November 2020 - December 2022).
\item Dolack, M.,
 MS. Computational morphogenesis of embryonic bone development: past, present, and future. (September 2017 - May 2019).
\item Gerber, J., 
MS. Development of a history-dependent damage model for the brain due to repetitive impacts. (November 2016 - May 2018).
\item Dhobale, A.,
 MS. Assessing functional connectivity of micro-tissue engineered neural networks using calcium fluorescence imaging. (August 2016 - May 2017).
\item Yuchi, L.,
 MS. A computational model of bidirectional growth for micro-Tissue Engineered Neuronal Networks (micro-TENNs). (August 2016 - May 2017).
\item Fang, Z.,
 MS. MPM methods for modeling trabecular bone. (August 2016 - May 2017).
\item Motiwale, S.,
 MS. Modeling intervertebral disc degeneration due to cyclic loading. (January 2015 - May 2016).
\item Ranslow, A.,
 MS. Microstructural analysis of porcine skull bone subjected to impact loading. Date Graduated: May 2016. (July 2014 - May 2016).
\item Fielding, R.,
 MS. Development of a lower extremity model for high strain rate impact loading. Date Graduated: May 2015. (September 2013 - May 2015).
\end{enumerate}

\subsubsection{Ph.D. Dissertation}\label{ph.d.-dissertation}

\begin{enumerate}
\def\labelenumi{\arabic{enumi}.}
\item
  Ph.D. Dissertation. (2023 - Present).\\
  Advised: Ryan Grube
\item
  Ph.D. Dissertation. (2022 - Present).\\
  Advised: Ann Reyes
\item
  Ph.D. Dissertation. (2017 - Present).\\
  Advised: Ritika Menghani
\item
  Ph.D. Dissertation. (2019 - August 2023).\\
  Advised: Valarie Martin
\item
  Ph.D. Dissertation. (January 2018 - July 2023).\\
  Advised: Thomas Hannah
\item
  Ph.D. Dissertation. (January 2015 - April 2023).\\
  Advised: Zacarie Hertel
\item
  Ph.D. Dissertation. (November 2015 - August 2020).\\
  Advised: Vikram Subramani
\item
  Ph.D. Dissertation. (December 2013 - May 2018).\\
  Advised: Chanyoung Lee
\item
  Ph.D. Dissertation. (September 2013 - June 2017).\\
  Advised: Harsha Garimella
\end{enumerate}

\subsubsection{Postdoctoral Mentorship}\label{postdoctoral-mentorship}

\begin{enumerate}
\def\labelenumi{\arabic{enumi}.}
\item
  Postdoctoral Mentorship. (September 2016 - July 2018).\\
  Advised: Toma Marinov
\end{enumerate}

\subsubsection{Undergraduate Honors
Thesis}\label{undergraduate-honors-thesis}

\begin{enumerate}
\def\labelenumi{\arabic{enumi}.}
\item
  Undergraduate Honors Thesis. (January 2021 - May 2022).\\
  Advised: Bennett Brown
\item
  Undergraduate Honors Thesis. (January 2021 - May 2022).\\
  Advised: Jaskson Mackay
\item
  Undergraduate Honors Thesis. (August 2018 - May 2021).\\
  Advised: Ouniol Aklilu
\item
  Undergraduate Honors Thesis. (August 2019 - April 2020).\\
  Advised: Lauren Katch
\item
  Undergraduate Honors Thesis. (November 2017 - May 2018).\\
  Advised: Patrick Casey
\item
  Undergraduate Honors Thesis. (January 2015 - May 2018).\\
  Advised: Patricia De Tomas-Medina
\item
  Undergraduate Honors Thesis. (August 2016 - May 2017).\\
  Advised: Marisa Borusiewicz
\item
  Undergraduate Honors Thesis. (September 2014 - May 2016).\\
  Advised: Kush Sodha
\item
  Undergraduate Honors Thesis. (September 2013 - May 2015).\\
  Advised: Michael Robinson
\end{enumerate}

\subsection{Teaching Experience}\label{teaching-experience}

\subsubsection{2024}\label{section}

ME 330, section 001, Computational Tools. 3 credit hours. 173 enrolled.

ME 330, section 001L, Computational Tools. 3 credit hours. 45 enrolled.

ME 330, section 002L, Computational Tools. 3 credit hours. 42 enrolled.

ME 330, section 003L, Computational Tools. 3 credit hours. 45 enrolled.

ME 330, section 004L, Computational Tools. 3 credit hours. 41 enrolled.

ME 563, section 001, Nonlin Finite Elem. 3 credit hours. 20 enrolled.

ME 563, section 001, Nonlin Finite Elem. 3 credit hours. 14 enrolled.

ME 596, section 002, Individual Studies. 3 credit hours. 1 enrolled.

ME 596, section 021, Individual Studies. Variable credit hours. 1
enrolled.

ME 600, section 021, Thesis Research. Variable credit hours. 1 enrolled.

\subsubsection{2023}\label{section-1}

ME 330, section 001, Computational Tools. 3 credit hours. 201 enrolled.

ME 330, section 001L, Computational Tools. 3 credit hours. 53 enrolled.

ME 330, section 002L, Computational Tools. 3 credit hours. 49 enrolled.

ME 330, section 003L, Computational Tools. 3 credit hours. 55 enrolled.

ME 330, section 004L, Computational Tools. 3 credit hours. 44 enrolled.

ME 596, section 027, Individual Studies. Variable credit hours. 3
enrolled.

ME 600, section 028, Thesis Research. Variable credit hours. 1 enrolled.

ME 360, section 001, Mechanical Design. 3 credit hours. 34 enrolled.

ME 461, section 001, Finite Elem Enger. 3 credit hours. 30 enrolled.

ME 461, section 001, Finite Elem Enger. 3 credit hours. 15 enrolled.

ME 600, section 027, Thesis Research. Variable credit hours. 2 enrolled.

ME 330, section 001, Computational Tools. 3 credit hours. 156 enrolled.

ME 330, section 001L, Computational Tools. 3 credit hours. 42 enrolled.

ME 330, section 002L, Computational Tools. 3 credit hours. 43 enrolled.

ME 330, section 003L, Computational Tools. 3 credit hours. 41 enrolled.

ME 330, section 004L, Computational Tools. 3 credit hours. 30 enrolled.

ME 600, section 021, Thesis Research. Variable credit hours. 1 enrolled.

\subsubsection{2022}\label{section-2}

ME 330, section 001, Computational Tools. 3 credit hours. 239 enrolled.

ME 330, section 001L, Computational Tools. 3 credit hours. 61 enrolled.

ME 330, section 002L, Computational Tools. 3 credit hours. 59 enrolled.

ME 330, section 003L, Computational Tools. 3 credit hours. 60 enrolled.

ME 330, section 004L, Computational Tools. 3 credit hours. 59 enrolled.

ME 600, section 027, Thesis Research. Variable credit hours. 2 enrolled.

ME 360, section 001, Mechanical Design. 3 credit hours. 40 enrolled.

ME 461, section 001, Finite Elem Enger. 3 credit hours. 39 enrolled.

ME 461, section 001, Finite Elem Enger. 3 credit hours. 16 enrolled.

ME 596, section 008, Individual Studies. 1 credit hours. 1 enrolled.

ME 494H, section 016, Senior Thesis. Variable credit hours. 2 enrolled.

ME 563, section 001, Nonlin Finite Elem. 3 credit hours. 11 enrolled.

ME 563, section 001, Nonlin Finite Elem. 3 credit hours. 9 enrolled.

ME 600, section 026, Thesis Research. Variable credit hours. 1 enrolled.

\subsubsection{2021}\label{section-3}

ME 330, section 001, Computational Tools. 3 credit hours. 220 enrolled.

ME 330, section 001L, Computational Tools. 3 credit hours. 30 enrolled.

ME 330, section 002L, Computational Tools. 3 credit hours. 30 enrolled.

ME 330, section 003L, Computational Tools. 3 credit hours. 32 enrolled.

ME 330, section 004L, Computational Tools. 3 credit hours. 32 enrolled.

ME 330, section 005L, Computational Tools. 3 credit hours. 32 enrolled.

ME 330, section 006L, Computational Tools. 3 credit hours. 32 enrolled.

ME 330, section 007L, Computational Tools. 3 credit hours. 32 enrolled.

ME 494H, section 022, Senior Thesis. Variable credit hours. 2 enrolled.

ME 596, section 003, Individual Studies. 3 credit hours. 1 enrolled.

ME 596, section 027, Individual Studies. Variable credit hours. 1
enrolled.

ME 600, section 027, Thesis Research. Variable credit hours. 2 enrolled.

ME 461, section 001, Finite Elem Enger. 3 credit hours. 42 enrolled.

ME 461, section 001, Finite Elem Enger. 3 credit hours. 4 enrolled.

ME 596, section 027, Individual Studies. Variable credit hours. 1
enrolled.

ME 600, section 027, Thesis Research. Variable credit hours. 1 enrolled.

ME 330, section 001, Computational Tools. 3 credit hours. 132 enrolled.

ME 330, section 002L, Computational Tools. 3 credit hours. 30 enrolled.

ME 330, section 003L, Computational Tools. 3 credit hours. 29 enrolled.

ME 330, section 004L, Computational Tools. 3 credit hours. 17 enrolled.

ME 330, section 005L, Computational Tools. 3 credit hours. 28 enrolled.

ME 330, section 006L, Computational Tools. 3 credit hours. 28 enrolled.

ME 596, section 026, Individual Studies. Variable credit hours. 1
enrolled.

ME 600, section 026, Thesis Research. Variable credit hours. 2 enrolled.

\subsubsection{2020}\label{section-4}

ME 330, section 001, Computational Tools. 3 credit hours. 177 enrolled.

ME 330, section 001L, Computational Tools. 3 credit hours. 29 enrolled.

ME 330, section 002L, Computational Tools. 3 credit hours. 30 enrolled.

ME 330, section 003L, Computational Tools. 3 credit hours. 31 enrolled.

ME 330, section 004L, Computational Tools. 3 credit hours. 27 enrolled.

ME 330, section 005L, Computational Tools. 3 credit hours. 30 enrolled.

ME 330, section 006L, Computational Tools. 3 credit hours. 30 enrolled.

ME 600, section 027, Thesis Research. Variable credit hours. 3 enrolled.

ME 461, section 001, Finite Elem Enger. 3 credit hours. 50 enrolled.

ME 461, section 001, Finite Elem Enger. 3 credit hours. 7 enrolled.

ME 497, section 004, Special Topics. 3 credit hours. 30 enrolled.

ME 563, section 001, Nonlin Finite Elem. 3 credit hours. 14 enrolled.

ME 563, section 001, Nonlin Finite Elem. 3 credit hours. 7 enrolled.

ME 596, section 026, Individual Studies. Variable credit hours. 2
enrolled.

ME 600, section 026, Thesis Research. Variable credit hours. 4 enrolled.

\subsubsection{2019}\label{section-5}

BME 496, section 019, Indep Studies. Variable credit hours. 1 enrolled.

ME 461, section 001, Finite Elem Enger. 3 credit hours. 18 enrolled.

ME 497, section 004, Special Topics. 3 credit hours. 28 enrolled.

ME 596, section 027, Individual Studies. Variable credit hours. 2
enrolled.

ME 600, section 027, Thesis Research. Variable credit hours. 2 enrolled.

ME 461, section 001, Finite Elem Enger. 3 credit hours. 20 enrolled.

ME 461, section 001, Finite Elem Enger. 3 credit hours. 27 enrolled.

ME 596, section 005, Individual Studies. 3 credit hours. 1 enrolled.

BME 496, section 018, Indep Studies. Variable credit hours. 2 enrolled.

ME 496, section 016, Indep Studies. Variable credit hours. 1 enrolled.

ME 563, section 001, Nonlin Finite Elem. 3 credit hours. 7 enrolled.

ME 563, section 001, Nonlin Finite Elem. 3 credit hours. 14 enrolled.

ME 596, section 026, Individual Studies. Variable credit hours. 1
enrolled.

ME 600, section 026, Thesis Research. Variable credit hours. 2 enrolled.

\subsubsection{2018}\label{section-6}

ME 461, section 001, Finite Elem Enger. 3 credit hours. 22 enrolled.

ME 461, section 002, Finite Elem Enger. 3 credit hours. 80 enrolled.

ME 497, section 004, Special Topics. 3 credit hours. 6 enrolled.

ME 596, section 022, Individual Studies. Variable credit hours. 1
enrolled.

ME 600, section 022, Thesis Research. Variable credit hours. 1 enrolled.

ME 461, section 001, Finite Elem Enger. 3 credit hours. 13 enrolled.

ME 461, section 001, Finite Elem Enger. 3 credit hours. 20 enrolled.

ME 600, section 022, Thesis Research. Variable credit hours. 1 enrolled.

ME 610, section 006, Thes Res Off Cmpus. Variable credit hours. 1
enrolled.

ME 494H, section 016, Senior Thesis. Variable credit hours. 1 enrolled.

ME 496, section 016, Indep Studies. Variable credit hours. 1 enrolled.

ME 497, section 003, Special Topics. 3 credit hours. 26 enrolled.

ME 497, section 004, Special Topics. 3 credit hours. 27 enrolled.

ME 563, section 001, Nonlin Finite Elem. 3 credit hours. 9 enrolled.

ME 563, section 001, Nonlin Finite Elem. 3 credit hours. 8 enrolled.

ME 596, section 022, Individual Studies. Variable credit hours. 1
enrolled.

ME 600, section 022, Thesis Research. Variable credit hours. 2 enrolled.

\subsubsection{2017}\label{section-7}

ME 461, section 001, Finite Elem Enger. 3 credit hours. 6 enrolled.

ME 461, section 002, Finite Elem Enger. 3 credit hours. 39 enrolled.

ME 461, section 004, Finite Elem Enger. 3 credit hours. 2 enrolled.

ME 494H, section 016, Senior Thesis. Variable credit hours. 2 enrolled.

ME 600, section 016, Thesis Research. Variable credit hours. 2 enrolled.

ME 461, section 001, Finite Elem Enger. 3 credit hours. 6 enrolled.

ME 461, section 001, Finite Elem Enger. 3 credit hours. 24 enrolled.

ME 596, section 004, Individual Studies. 3 credit hours. 1 enrolled.

ME 494H, section 017, Senior Thesis. Variable credit hours. 1 enrolled.

ME 496, section 016, Indep Studies. Variable credit hours. 3 enrolled.

ME 563, section 001, Nonlin Finite Elem. 3 credit hours. 14 enrolled.

ME 596, section 016, Individual Studies. Variable credit hours. 5
enrolled.

\subsubsection{2016}\label{section-8}

ME 440, section 005, Mech Sys Design. 3 credit hours. 14 enrolled.

ME 494, section 016, Research Project. Variable credit hours. 1
enrolled.

ME 496, section 016, Indep Studies. Variable credit hours. 3 enrolled.

ME 596, section 017, Individual Studies. Variable credit hours. 6
enrolled.

M E 600, section 018, Thesis Research. 1 credit hours. 2 enrolled.

M E 610, section 001, Thesis Research Off Campus. 1 credit hours. 6
enrolled.

M E 461, section 001, Finite Elements in Engineering. 3 credit hours. 32
enrolled.

M E 494H, section 016, Senior Thesis. 1 credit hours. 1 enrolled.

M E 563, section 001, Nonlinear Finite Elements. 3 credit hours. 14
enrolled.

M E 596, section 017, Individual Studies. 1 credit hours. 6 enrolled.

M E 600, section 017, Thesis Research. Variable credit hours. 1
enrolled.

\subsubsection{2015}\label{section-9}

M E 461, section 002, Finite Elements in Engineering. 3 credit hours. 35
enrolled.

M E 494H, section 014, Senior Thesis. 2 credit hours. 1 enrolled.

M E 596, section 013, Individual Studies. 4 credit hours. 3 enrolled.

M E 600, section 013, Thesis Research. 6 credit hours. 1 enrolled.

M E 494H, section 014, Senior Thesis. 4 credit hours. 2 enrolled.

M E 496, section 014, Independent Studies. 2 credit hours. 2 enrolled.

M E 563, section 001, Nonlinear Finite Elements. 3 credit hours. 13
enrolled.

M E 596, section 014, Individual Studies. 6 credit hours. 2 enrolled.

M E 600, section 014, Thesis Research. 3 credit hours. 1 enrolled.

ME 461, section 001, Finite Elem Enger. 3 credit hours. 37 enrolled.

\subsubsection{2014}\label{section-10}

M E 360, section 002, Mechanical Design. 3 credit hours. 94 enrolled.

M E 496, section 011, Independent Studies. 3 credit hours. 1 enrolled.

M E 596, section 016, Individual Studies. 4.5 credit hours. 2 enrolled.

M E 496, section 018, Independent Studies. 2 credit hours. 2 enrolled.

M E 496, section 024, Independent Studies. 3 credit hours. 1 enrolled.

M E 563, section 001, Nonlinear Finite Elements. 3 credit hours. 9
enrolled.

\subsubsection{2013}\label{section-11}

M E 360, section 002, Mechanical Design. 3 credit hours. 83 enrolled.

M E 496, section 047, Independent Studies. 1.7 credit hours. 3 enrolled.

\subsection{Service}\label{service}

\subsubsection{College}\label{college}

College, Committee Work, Member, "Engineering laptop Initiative". (July
2021 - December 2021).

College, Committee Work, Member, "Activity Insight Faculty Users
Committee". (October 2017 - December 2020).

College, Academic Leadership and Support Work, Member, "College of
Engineering National Science Foundation CAREER Award Winners". (April
2016).

College, Competition Judging, Judge, "College of Engineering Symposium
for Undergraduate Research", College of Engineering Symposium for
Undergraduate Research. (April 2014).

\subsubsection{Department}\label{department}

Department, Committee Work, Committee Member, "Mechanical Engineering
Promotion and Tenure Committee". (July 2022 - Present).

Department, Academic Leadership and Support Work, Representative,
"Biomechanics \& Biodevices Research Supergroup Department Lead".
(August 2023 - May 2024).

Department, Committee Work, Chairperson, "Mechanical Engineering
Strategic Plan Tracking Committee". (August 2023 - May 2024).

Department, Committee Work, Member, "Promotion and Tenure Committee".
(August 2023 - May 2024).

Department, Committee Work, Member, "Research Advancement Committee".
(August 2023 - May 2024).

Department, Committee Work, Chairperson, "Teaching Load Policy
Committee". (September 2022 - May 2023).

Department, Academic Leadership and Support Work, Representative,
"Biomechanics \& Biodevices Research Supergroup Department Lead".
(August 2022 - May 2023).

Department, Committee Work, Member, "Department Facilities Committee".
(August 2022 - May 2023).

Department, Committee Work, Member, "Promotion and Tenure Committee".
(August 2022 - May 2023).

Department, Committee Work, Member, "Research Advancement Committee".
(August 2022 - May 2023).

Department, Committee Work, Committee Member, "Mechanical Engineering
Strategic Planning Committee". (January 2022 - December 2022).

Department, Committee Work, Chairperson, "Teaching Load Policy
Committee". (September 2020 - May 2021).

Department, Committee Work, Member, "Mechanical Engineering Strategic
Planning Committee". (August 2019 - September 2020).

Department, Committee Work, Chairperson, "Joint Faculty Search in
Mechanical Engineering and the Institute for CyberScience". (August 2018
- May 2019).

Department, Committee Work, Member, "Faculty Search Committee for
Mechanical Systems in Mechanical Engineering". (August 2017 - 2018).

Department, Committee Work, Liaison, "Mechanical Engineering Liaison to
Institute for CyberScience". (2017 - 2018).

Department, Committee Work, Member, "Faculty Search Committee for
Emerging Areas in Mechanical Engineering". (August 2016 - 2017).

Department, Committee Work, Member, "Faculty Search Committee for
Mechanical Systems in Mechanical Engineering", Search Committee for
Mechanical Systems in Mechanical Engineering. (September 2014 - March
2015).

\subsubsection{University}\label{university}

University, Committee Work, Chairperson, "Institute for Computational \&
Data Sciences Coordinating Committee". (August 2023 - May 2024).

University, Committee Work, Committee Member, "Graduate Council
Committee on Academic Standards", Graduate Council Committee on Academic
Standards. (June 2023 - May 2024).

University, Committee Work, Committee Member, "Graduate Council
Representative to Engineering Faculty". (June 2023 - May 2024).

University, Committee Work, Co-Chairperson, "Institute for Computational
\& Data Sciences Coordinating Committee". (August 2022 - May 2023).

University, Participation in Development/Fundraising Activities, Member,
"AI/ML Faculty Engagement Team on behalf of Institute for CyberScience".
(October 2019 - May 2020).

University, Participation in Development/Fundraising Activities, Member,
"Faculty Participant", Coalition team sent to IBM research headquarters
on behalf of Institute for CyberScience. (April 2019).

University, Committee Work, Member, "Hiring Committee for Project
Coordinator for Institute for CyberScience". (March 2019 - April 2019).

\subsubsection{Profession}\label{profession}

Profession, Organizing Conferences and Service on Conference Committees,
"Steering Committee Member (Elected)", American Society of Mechanical
Engineering International Mechanical Engineering Congress and Exposition
(IMECE). (2021 - Present).

Yes, Profession, Organizing Conferences and Service on Conference
Committees, "Technical Chair (Elected)", American Society of Mechanical
Engineering International Mechanical Engineering Congress and Exposition
(IMECE). (November 2023 - November 2024).

Yes, Profession, Organizing Conferences and Service on Conference
Committees, Co-Organizer, Co-Organizer, "Organizer for Biological and
Biomimetic Soft Materials Symposium", 2024 Mach Conference. (April
2024).

Yes, Profession, Organizing Conferences and Service on Conference
Committees, "Vice Technical Chair (Elected)", American Society of
Mechanical Engineering International Mechanical Engineering Congress and
Exposition (IMECE). (November 2022 - November 2023).

Profession, Organizing Conferences and Service on Conference Committees,
Co-Chairperson, Co-Chairperson, "Chair of Brain and Injury Mechanics
Symposium", Brain and Injury Mechanics Symposium, SB3C Conference, Vail,
Colorado. (June 2023).

Yes, Profession, Organizing Conferences and Service on Conference
Committees, Co-Organizer, Co-Organizer, "Organizer for Biological and
Biomimetic Soft Materials Symposium", 2023 Mach Conference. (April
2023).

Profession, Organizing Conferences and Service on Conference Committees,
"Primary Organizer and Co-Chairperson", Damage Biomechanics Symposium at
the 2022 ASME International Mechanical Engineering Congress and
Exposition (IMECE). (November 2021 - November 2022).

Profession, Organizing Conferences and Service on Conference Committees,
"Track Co-Chair", Biomedical \& Biotechnology Engineering Track at the
2022 ASME International Mechanical Engineering Congress and Exposition
(IMECE). (November 2021 - November 2022).

Profession, Organizing Conferences and Service on Conference Committees,
Co-Organizer, Co-Organizer, "Organizer for Injury Biomechanics
Symposium", 2022 Society of Engineering Science (SES) Annual Technical
Meeting. (February 2022 - October 2022).

Profession, Organizing Conferences and Service on Conference Committees,
Co-Organizer, Co-Organizer, "Primary Organizer and Co-Chairperson",
Damage Biomechanics Symposium at the 2021 ASME International Mechanical
Engineering Congress and Exposition (IMECE). (November 2020 - November
2021).

Profession, Organizing Conferences and Service on Conference Committees,
Co-Organizer, Co-Organizer, "Primary Organizer and Co-Chairperson",
Damage Biomechanics Symposium at the 2020 ASME International Mechanical
Engineering Congress and Exposition (IMECE). (November 2019 - November
2020).

Profession, Organizing Conferences and Service on Conference Committees,
Co-Organizer, Co-Organizer, "Primary Organizer and Co-Chairperson",
Damage Biomechanics Symposium at the 2019 ASME International Mechanical
Engineering Congress and Exposition (IMECE). (November 2018 - November
2019).

Profession, Organizing Conferences and Service on Conference Committees,
Co-Chairperson, Co-Chairperson, "Chair of Brain Biomechanics II -
Measurement and modeling Symposium", Brain Biomechanics II - Measurement
and modeling Symposium, New York City, New York, U.S.A. (August 2019).

Profession, Organizing Conferences and Service on Conference Committees,
Co-Chairperson, Co-Chairperson, "Chair of Growth Remodeling and Repair
II: Musculoskeletal System Symposium", Summer Biomechanics,
Bioengineering, and Biotransport (SB3C) Conference, Seven Springs,
Pennsylvania, U.S.A. (June 2019).

Profession, Organizing Conferences and Service on Conference Committees,
Co-Chairperson, Co-Chairperson, "Primary Organizer and Co-Chairperson",
Special symposium on "Computational Modeling of Morphogenesis: Friend or
Foe?" at the annual meeting of American Association of Anatomists (AAA),
Orlando, Florida. (May 2018 - April 2019).

Profession, Organizing Conferences and Service on Conference Committees,
Co-Organizer, Co-Organizer, "Primary Organizer and Co-Chairperson",
Damage Biomechanics Symposium at 2018 ASME International Mechanical
Engineering Congress and Exposition (IMECE). (November 2017 - November
2018).

Profession, Organizing Conferences and Service on Conference Committees,
Co-Organizer, Co-Organizer, "Co-Organizer", "Multiscale Brain Mechanics:
From Growth to Injury" Symposium at 18th U.S. National Congress for
Theoretical and Applied Mechanics. (August 2017 - June 2018).

Profession, Organizing Conferences and Service on Conference Committees,
Co-Organizer, Co-Organizer, "Primary Organizer and Co-Chairperson",
Damage Biomechanics Symposium at the 2017 ASME International Mechanical
Engineering Congress and Exposition (IMECE). (November 2016 - November
2017).

Profession, Organizing Conferences and Service on Conference Committees,
Co-Organizer, Co-Organizer, "Primary Organizer and Co-Chairperson",
Damage Biomechanics Symposium at the 2016 ASME International Mechanical
Engineering Congress and Exposition (IMECE). (November 2015 - November
2016).

Profession, Organizing Conferences and Service on Conference Committees,
Co-Chairperson, Co-Chairperson, "Co-Chairperson", Brain Injury Symposium
at Summer Biomechanics, Bioengineering, and Biotransport (SB3C)
Conference, National Harbor, Maryland, U.S.A. (June 2016).

Profession, Organizing Conferences and Service on Conference Committees,
Co-Organizer, Co-Organizer, "Co-Organizer and Co-Chairperson", Damage
Biomechanics Symposium at the 2015 ASME International Mechanical
Engineering Congress and Exposition (IMECE), Houston, Texas, USA, Damage
Biomechanics. (December 2014 - November 2015).

Profession, Organizing Conferences and Service on Conference Committees,
Co-Organizer, Co-Organizer, "Co-Organizer and Co-Chairperson", Advances
in Computational Biomechanics Symposium at 2015 Pan-American Congress on
Computational Mechanics International Conference, Buenos Aires,,
Argentina. (June 2014 - June 2015).

Profession, Organizing Conferences and Service on Conference Committees,
Co-Organizer, Co-Organizer, "Co-Organizer and Co-Chairperson", 2014
Mid-Atlantic Section (M-AS) of the American Physical Society (APS),
University Park, Pennsylvania, USA. (January 2014 - October 2014).

\subsubsection{Society}\label{society}

Society, Service to Governmental Agencies, Panelist, Panelist, "NDSEG
Fellowship Evaluation Panelist", National Defense Science and
Engineering Graduate (NDSEG) Fellowship program. (2015).

\subsection{Editorial Board Positions:}\label{editorial-board-positions}

ASME Journal of Engineering and Science in Medical Diagnostics and
Therapy (JESMDT), Associate Editor. (November 2022 - Present).

Frontiers in Bioengineering and Biotechnology, Associate Editor.
(November 2014 - Present).

\subsection{Professional Memberships}\label{professional-memberships}

Member, American Society of Mechanical Engineers. (January 2003 -
Present).

Member, United States Association for Computational Mechanics. (February
2014 - 2015).

Member, American Physical Society. (May 2013 - 2014).

Member, American Society for Engineering Education. (May 2013 - 2014).

Member, American Society of Biomechanics. (May 2013 - 2014).



\subsection{Ph.D. Dissertation Advisor}\label{phd-dissertation-advisor}

\begin{enumerate}
  \def\labelenumi{\arabic{enumi}.}
    \item Grube, R.,
     Ph.D. High strain rate material response of Dyneema. (2023 - Present). Pre-candidacy.
    \item Reyes, A.,
     Ph.D. Modeling of spinal disc degeneration in fighter jet pilots. (2022 - Present). Post-candidacy.
    \item Menghani, R.,
     Ph.D. Sensor enabled, cloud-based modeling of the brain. (2017 - Present). Post-comprehensive. 2024 Marcus Engineering Research Fellowship.
    \item Martin, V.,
     Ph.D. Modeling Armor Composites Undergoing High Strain Rate Deformation. (2019 - August 2023).
    \item Hannah, T.,
     Ph.D. Computational and experimental characterization of high strain rate response of Dyneema. (January 2018 - July 2023).
    \item Hertel, Z.,
     Ph.D. An exploration of the Material Point Method (MPM) in CTH applied to soft material systems subjected to dynamic loading. (January 2015 - April 2023).
    \item Subramani, V.,
     Ph.D. Modeling of spinal injury under extreme loading Conditions with emphasis on military loading Scenarios – a mathematical fatigue damage model and finite element study. (November 2015 - August 2020).
    \item Lee, C., Ph.D.
     A computational analysis of bone formation in the cranial vault using a reaction-diffusion-strain model. (December 2013 - May 2018).
    \item Garimella, H.,
     Ph.D. An embedded element based human head model to investigate axonal injury. (September 2013 - June 2017).
\end{enumerate}

\subsection{Ph.D. Dissertation Committee Member}\label{phd-dissertation-committee-member}

\begin{enumerate}
  \def\labelenumi{\arabic{enumi}.}
    \item Tugba, H. A novel treatment for facet joint pain using radiofrequency ablation. (July 2021 - October 2022). Advisor: D. Cortes
    \item Young, J. Steady-state response of mechanical power flow to structural modifications. (August 2021 - February 2022). Advisor: R. Campbell
    \item Damirchi, B. Computational investigation on carbon nanotube – composite interactions using the ReaxFF reactive force field. (March 2019 - April 2021). Advisor: Adri Van Duin
    \item Gauntt, S. Dynamics of hybrid gears as part of VLRCOE. (February 2019 - February 2021). Advisor: Rob Campbell and Sean McIntyre
    \item Patki, P. Modeling and computational of bio-degradation in engineered tissue scaffolds. (September 2017 - December 2020). Advisor: F. Costanzo
    \item Zhou, Y. 3D multiscale bone biomechanics study: Effect of disease and treatment. (January 2018 - November 2020). Advisor: J. Du
    \item Rezwan, A. Evaluation of a multi-metallic layered composite fuel cladding for improved accident tolerance using multiscale modeling and simulation. (June 2017 - December 2019). Advisor: M. Tonks
    \item Hudson, R., Ph.D. Computational method for modeling the vibrational properties of Nanocomposities with Embedded Carbon Nanotubes. (August 2016 - June 2018). Advisor: A. Sinha.
    \item Treacy, S. Stability analysis and experimental testing of fluidic pitch links in helicopters with articulated rotors. (November 2016 - July 2017). Advisor: C. Rahn
    \item Ma, Z. Understanding brain networks in rats and humans: Data mining in neuroimaging. (February 2017 - June 2017). Advisor: N. Zhang
    \item Wang, B., Ph.D. Effects of external stimuli on microstructure-property relationship at the nanoscale. (August 2014 - June 2017). Advisor: A. Haque.
    \item Gouge, M., Ph.D. Advancements in thermo-mechanical model development and experimental validation for direct deposition additive manufacturing processes. (December 2014 - February 2016). Advisor: P. Michaleris.
    \item Denlinger, E., Ph.D. Thermo-mechanical model development and experimental validation for metallic parts in additive manufacturing. (October 2014 - June 2015). Advisor: P. Michaleris.
\end{enumerate}

\subsection{Postdoctoral Mentorship Advisor}\label{postdoctoral-mentorship-advisor}

\begin{enumerate}
  \def\labelenumi{\arabic{enumi}.}
    \item Marinov, T. Computational neuroscience: simulation of micro-tissue engineered neural networks. (September 2016 - July 2018).
\end{enumerate}

\subsection{Research Activity Advisor}\label{research-activity-advisor}

\begin{enumerate}
  \def\labelenumi{\arabic{enumi}.}
    \item Caponi, L., Undergraduate. Imaging and modeling associated with split-hopkinson pressure bar testing. (June 2018 - August 2018). Toshiba Westinghouse Summer Fellowship Program.
    \item McDonough, B., Undergraduate. Investigation of shear thickening fluids for personal armor. (December 2015 - May 2016). College of Engineering Research Initiative (CERI)
    \item Kozuch, C., Undergraduate. Modeling dynamic fracture in bones. (September 2013 - May 2016).
    \item Catherman, B., Undergraduate. Developing a miniaturized Kolsky bar for high strain rate mechanical testing of soft tissues. (May 2014 - December 2015).
    \item Shannon, R., Undergraduate. Developing algorithms for creating statistical material properties meshes for bone. (January 2014 - December 2015).
    \item Ho, C., Undergraduate. Scalable, fast algorithms for wireless biomechanical sensors. (January 2015 - May 2015).
    \item Zhang, Y., Undergraduate. Finite element simulations of intervertebral discs. (August 2014 - December 2014).
    \item Lukens, P., Undergraduate. Measuring head and neck biomechanics in sports. (June 2014 - December 2014).
    \item de Oliveira Pereira, D., Undergraduate. Novel designs of combat boots. (May 2014 - December 2014). Exchange Student from Brazil.
    \item Yuan, H., Undergraduate. From pictures to parallel computing: Making an anatomic finite element model. (September 2013 - August 2014). 2014 Penn State College of Engineering Research Experience for Undergraduates Fellowship.
    \item Roudabush, E., Undergraduate. Exploring the computer science of finite elements. (September 2013 - May 2014).
    \item Putnam, H., Undergraduate. 3D printing a calcaneus and anatomic measurements. (September 2013 - May 2014).
    \item McGoldrick, M., Undergraduate. Exploring intersections of biology and engineering. (September 2013 - May 2014).
\end{enumerate}

\subsection{Undergraduate Honors Thesis Advisor}\label{undergraduate-honors-thesis-advisor}

\begin{enumerate}
  \def\labelenumi{\arabic{enumi}.}
    \item Brown, B.,
     Undergraduate. Advanced visualization techniques for brain modeling. (January 2021 - May 2022). Schreyer's Honors College.
    \item Mackay, J.,
     Undergraduate. Brain impact analysis from overpressure sources through machine learning based on explosion simulations and wearable blast gauges. (January 2021 - May 2022). Schreyer's Honors College.
    \item Aklilu, O.,
     Undergraduate. Experimental and computational investigation of correlates of diffusion tensor imaging changes and mechanical strain. Date Graduated: May 2021. (August 2018 - May 2021). Millennium Scholars Program and Schreyer Honors Student.
    \item Katch, L.,
     Undergraduate. Reverse source localization for identification of overpressure sources based on wearable blast gauges. (August 2019 - April 2020).
    \item Casey, P.,
     Undergraduate. Behavior of a modeled hip implant insertion device through finite element analysis. (November 2017 - May 2018). Schreyer's Honor College.
    \item De Tomas-Medina, P.,
     Undergraduate. Modeling the response of neurons subjected to high rate deformation: Comparing simulations to experimental results. (January 2015 - May 2018). Millennium Scholars Program. Schreyer's Honor College.
    \item Borusiewicz, M.,
     Undergraduate. Quantifying the structure of micro-tissue engineered neural networks. (August 2016 - May 2017). Schreyer's Honor College.
    \item Sodha, K.,
     Undergraduate. Estimating dynamic properties for biological materials: Design, development, and calibration of a desktop miniaturized double-lap shear Kolsky bar. Date Graduated: May 2016. (September 2014 - May 2016). Schreyer's Honors College.
    \item Robinson, M.,
     Undergraduate. The development of an anatomically correct model of calcaneus fracture and fragmentation due to impact loading. Date Graduated: May 2015. (September 2013 - May 2015). Schreyer's Honors College.
\end{enumerate}

\end{document}
